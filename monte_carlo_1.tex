\chapter{Monte Carlo I}

% -----------------------------------------------------------------------------
\section{Historical Background}

\subsection{Monte Carlo Method}


\emph{Early Days}

  \begin{itemize}
  \item Jacob Bernoulli [1689]
  \item George-Louis Leclerc (Comte de Buffon) [1777]
  \item Pierre-Simon de Laplace [1786]
  \item Lord Rayleigh [1899]
  \item Fermi [1930's]
  \item Metropolis and Ulam [1946+]
  \end{itemize}



\emph{Early Days}

  \begin{figure}[htbp]
    \centering
    %\includegraphics[keepaspectratio, width = 3.2 in]{images/metropolis_and_ulam.jpg}
  \end{figure}



\emph{Applications to Neutron Diffusion}
 \begin{itemize}
  \item Von Neumann quickly saw the possibilities of digital computing and statistical sampling to explore the behavior of fission chain reactions.
  \begin{itemize}
   \item He outlined the initial details for static neutron diffusion calculations in 1947
   \item He layed out a sampling scheme using spherical geometry and known nuclear properties.
   \item He calculated that following 100 neutrons through 100 collisions would take about 5 hours!
   \item He also proposed a way to treat coupled radiation hydrodynamics problems using a non-linear iterative process.
  \end{itemize}
 \end{itemize}


\subsection{Simple Example}
\emph{Two Approaches to Monte Carlo}
 Evaluate $G=\int_{0}^{1} g(x)dx$, with $g(x) = \sqrt{1-x^2}$
 \begin{itemize}
   \item Simulation approach
   \item Mathematical approach (aka ``Quadrature'')
 \end{itemize}
 \vfill
 A \emph{ third} approach: Markov Chain Monte Carlo (MCMC)


\emph{Monte Carlo Neutron Transport - In a nutshell}
 \begin{itemize}
  \item Simulate the stochasticity of nature by tracking neutrons randomly through our domain of interest from their birth to their death
  \item Use nuclear data and material properties to determine probabilities of various events
  \item Evaluate contribution of individual neutrons to a quantity of interest
  \item In addition to common modeling and data uncertainties, Monte Carlo results always present an element of statistical uncertainty
  \begin{itemize}
    \item Average is never enough, you must always provide a measure of dispersion.
  \end{itemize}
 \end{itemize}


\emph{Topics to be covered}
   \begin{itemize}
      \item Probability and Statistics (Today)
      \item Random Number Generators, Sampling, Collision Physics
      \item Tallies and Statistics
      \item Variance Reduction
   \end{itemize}


% -----------------------------------------------------------------------------
\section{Probability}

\subsection{Basic Definitions}

\emph{Sample Space}

 \begin{itemize}
  \item Set of all possible outcomes of an experiment is called the sample space
  \begin{itemize}
    \item Coin toss \{Heads,Tails\}
    \item Roll of a die \{1,2,3,4,5,6\}
    \item Continuous sample spaces have an infinite number of outcomes
  \end{itemize}
  \item A subset of a sample space is called an event
 \end{itemize}



\emph{Probabilities}
 Each event in a sample space has a probability of occurring
 \begin{itemize}
  \item If $S$ is the sample space, then $P(S) = 1$
  \item For any event $A$, $0 \le P(A) \le 1$
 \end{itemize}



\subsection{Random Variables}
\emph{Random Variables}
  A random variable assigns a numerical value to each outcome in a sample space.  Denoted with uppercase letters: $X, Y, Z$
  \begin{itemize}
    \item It's a function from the sample space to the real line
      \begin{equation*}
        X: S \to \mathbb{R}
      \end{equation*}
     \item Example
      \begin{itemize}
       \item Flip two coins
         \begin{equation*}
            S = \{HH,HT,TH,TT\}
         \end{equation*}
        Suppose \textit{X} is the random variable corresponding to the number of \textit{T}'s
         \begin{equation*}
           X(TT)=2, X(HT)=X(TH)=1, X(HH)=0
         \end{equation*}
      \end{itemize}
  \end{itemize}


\emph{Discrete Random Variables}
 A random variable is discrete if its possible values form a discrete set.
%  Example in neutron transport
     \begin{itemize}
      \item At a given neutron energy and collision site in a homogeneous material $(B_{4}C)$, sample the collision nuclide
     \end{itemize}
     \begin{equation*}
         \Sigma_{B} = 0.3 cm^{-1}
     \end{equation*}
     \begin{equation*}
         \Sigma_{C} = 0.1 cm^{-1}
     \end{equation*}
     Define
     \begin{displaymath}
        X = \left\{ \begin{array}{ll}
                     0 & \textrm{if $C$ is selected}\\
                     1 & \textrm{if $B$ is selected}
                    \end{array} \right.
     \end{displaymath}


\emph{Probability Mass Function}
 The probablity mass function (pmf) of a discrete random variable $X$ is the function $p(x) \equiv P(X=x)$.
     \begin{itemize}
      \item Note that $p(x) \ge 0$
      \item $\sum_{x} p(x) = 1$
      \item also implies $p(x) \le 1$
     \end{itemize}
     From previous example
     \begin{displaymath}
        p(x) = P(X=x) = \left\{ \begin{array}{ll}
                         0.25 & \textrm{if $x=0$}\\                       
                         0.75 & \textrm{if $x=1$}\\
                         0 & \textrm{otherwise}
                       \end{array} \right.
     \end{displaymath}
%     \begin{itemize}
%      \item Probability of selecting $B: P(B) = 0.75$
%      \item Probability of selecting $C: P(C) = 0.25$
%     \end{itemize}

% Insert pmf of previous example
%     \begin{displaymath}
%        p(x) = \left\{ \begin{array}{ll}
%                        0.75 & \textrm{if }
%                       \end{array}

%     \end{displaymath}



\emph{Continuous Random Variables}
 A continuous random variable is one with probability of 0 at every point.
\begin{itemize}
 \item Example: Select a real number between 0 and 1.
   \begin{itemize}
    \item Infinite number of equally probable events
    \item $P($at each point$) = P(X=x) = 0$
    \item However, $P(X \le 0.5) = 0.5$
   \end{itemize}
 \item When using continuous variables we always deal with intervals
\end{itemize}


\emph{Probability Distribution Function}
 If $X$ is a continuous random variable, $p(x)$ is a probability distribution function if:
 \begin{itemize}
  \item $\int_{\mathbb{R}} p(x)dx = 1$
  \item $p(x) \ge 0, \forall x$
  \item If $A \subseteq \mathbb{R}$, then $P(X \in A) = \int_{A}p(x)dx$
 \end{itemize}


\emph{Continuous pdf}
 The probability distribution function describing the path travel by a neutral particle can be expressed by:
     \begin{displaymath}
        p(x) = \left\{ \begin{array}{ll}
                         \Sigma_{t}e^{-\Sigma_{t}x} & \textrm{if $x\ge0$}\\                       
                         0 & \textrm{otherwise}
                       \end{array} \right.
     \end{displaymath}
This distribution is also called the exponential distribution. $(X \sim Exp(\Sigma))$


\emph{Example of Exponential Distribution}
  Suppose $X \sim Exp(1)$ (Comparable to a thermal neutron in water)
  \begin{itemize}
    \item $P(X\le1)=\int_{0}^{1}e^{-x}dx=1-e^{-1} = 0.632$
    \item $P(X\le3)=\int_{0}^{3}e^{-x}dx=1-e^{-3} = 0.950$
    \item $P(X\ge5)=\int_{5}^{\infty}e^{-x}dx=e^{-5}=0.0067$
    \item $P(X=3)=\int_{3}^{3}e^{-x}dx=0$
  \end{itemize}
 This distribution defines the distance traveled by a neutron between collisions.


\emph{Uniform Distribution}
  If $X$ is equally likely to be anywhere between $a$ and $b$, then $X$ has the uniform distribution on $(a,b)$
     \begin{displaymath}
        f(x) = \left\{ \begin{array}{ll}
                         \frac{1}{(b-a)} & \textrm{if $a < x < b$}\\                       
                         0 & \textrm{otherwise}
                       \end{array} \right.
     \end{displaymath}
  Notation $X \sim U(a,b)$\\
  Very important distribution for sampling sources over a uniform volume.


\subsection{Distribution Functions}

\emph{Cumulative Distribution Function}
  For any random variable $X$, the cumulative distribution function is defined for all $x$ by $F(x) = P(X\le x))$
  \begin{itemize}
    \item $X$ continuous implies
      \begin{equation*}
          F(x) = \int_{-\infty}^{x}f(t)dt
      \end{equation*}
    \item $X$ discrete implies
      \begin{equation*}
          F(x) = \sum_{y\le x}f(y)
      \end{equation*}
  \end{itemize}


\emph{Continuous cdf}
  Theorem: If $X$ is a continuous random variable, then $f(x) = F'(x)$


\emph{Uniform Distribution cdf}
  Example: $X \sim U(0,1)$
     \begin{displaymath}
        f(x) = \left\{ \begin{array}{ll}
                         1 & \textrm{if $0 < x < 1$}\\                       
                         0 & \textrm{otherwise}
                       \end{array} \right.
     \end{displaymath}
     \begin{displaymath}
        F(x) = \left\{ \begin{array}{ll}
                         0 & \textrm{if $x \le 0$}\\                       
                         x & \textrm{if $0 < x < 1$}\\
                         1 & \textrm{if $x > 1$}
                       \end{array} \right.
     \end{displaymath}
  This distribution is the most commonly used in any Monte Carlo simulation.


\emph{Exponential Distribution cdf}
  Example: $X \sim Exp(\Sigma)$
     \begin{displaymath}
        f(x) = \left\{ \begin{array}{ll}
                         \Sigma e^{-\Sigma x} & \textrm{if $x \ge 0$}\\                       
                         0 & \textrm{otherwise}
                       \end{array} \right.
     \end{displaymath}
     \begin{displaymath}
        F(x) = \int_{-\infty}^{x}f(t)dt = \left\{ \begin{array}{ll}
                         0 & \textrm{if $x < 0$}\\                       
                         1-e^{-\Sigma x} & \textrm{if $x \ge 0$}
                       \end{array} \right.
     \end{displaymath}


\emph{Discrete cdf}
% Put nuclide sampling example and/or reaction sampling example
 Example: Nuclide sampling - Borated Steel
  \begin{itemize}
    \item Convert material properties to individual nuclide densities
  \end{itemize}
     \begin{displaymath}
        \left. \begin{array}{ll}
               Fe^{54}  & 6.32x10^{21} \textrm{atoms/cm3}\\
               Fe^{56}  & 7.44x10^{22} \textrm{atoms/cm3}\\
               Ni^{58}  & 1.67x10^{22} \textrm{atoms/cm3}\\
               Ni^{60}  & 8.85x10^{21} \textrm{atoms/cm3}\\
               B^{10}  & 3.24x10^{20} \textrm{atoms/cm3}\\
               B^{11}  & 1.21x10^{21} \textrm{atoms/cm3}
                       \end{array} \right.
  \end{displaymath}


\emph{Discrete cdf (2)}
  \begin{itemize}
    \item For a given incoming neutron energy, evaluate microscopic total cross-section
    \item Calculate total macroscopic cross-section
  \end{itemize}
     \begin{displaymath}
        \left. \begin{array}{llll}
               Fe^{54}  & 4 \: \textrm{barns} & 0.02528 & cm^{-1}\\
               Fe^{56}  & 10 \: \textrm{barns} &  0.744 & cm^{-1}\\
               Ni^{58}  & 14 \: \textrm{barns} & 0.2338 & cm^{-1}\\
               Ni^{60}  & 1 \: \textrm{barns} & 0.00885 & cm^{-1}\\
               B^{10}  & 1000 \: \textrm{barns} & 0.324 & cm^{-1}\\
               B^{11}  & 7 \: \textrm{barns} & 0.00847 & cm^{-1}
                       \end{array} \right.
  \end{displaymath}  


\emph{Discrete cdf (3)}
  \begin{itemize}
    \item Total macroscopic cross-section is 1.3444 $cm^{-1}$.
    \item Normalize to 1 to create pmf
  \end{itemize}
     \begin{displaymath}
        f(x) = \left\{ \begin{array}{ll}
                         0.0188 & \textrm{if $x = 1$}\\                       
                         0.5534 & \textrm{if $x = 2$}\\
                         0.1739 & \textrm{if $x = 3$}\\
                         0.0066 & \textrm{if $x = 4$}\\
                         0.2410 & \textrm{if $x = 5$}\\
                         0.0063 & \textrm{if $x = 6$}\\
                         0 & \textrm{otherwise}
                       \end{array} \right.
     \end{displaymath}


\emph{Discrete cdf (4)}
  \begin{itemize}
    \item Convert to cdf 
  \end{itemize}
     \begin{displaymath}
        F(x) = \left\{ \begin{array}{ll}
                         0.0    & \textrm{if $x \le 0$}\\                       
                         0.0188 & \textrm{if $0 < x \le 1$}\\
                         0.5722 & \textrm{if $1 < x \le 2$}\\
                         0.7461 & \textrm{if $2 < x \le 3$}\\
                         0.7527 & \textrm{if $3 < x \le 4$}\\
                         0.9937 & \textrm{if $4 < x \le 5$}\\
                         1.0 & \textrm{if $x > 5$}
                       \end{array} \right.
     \end{displaymath}
  \begin{itemize}
    \item Notice that the discrete pmf is now converted to a continuous distribution
    \item Pay attention to the $<$ and the $\le$  
  \end{itemize}


\emph{Another Example}
  Assume a neutron of 10MeV will collide with $Fe^{56}$
  \begin{itemize}
    \item Evaluate all possible cross-section at that energy
  \end{itemize}
     \begin{displaymath}
        \left. \begin{array}{ll}
               (n,n) & 2.0 \: \textrm{barns}\\
               (n,n') & 0.1 \: \textrm{barns}\\
               (n,\gamma) & 0.001 \: \textrm{barns}\\
               (n,\alpha) & 0.1 \: \textrm{barns}
               \end{array} \right.
  \end{displaymath} 
  \begin{itemize}
    \item Total cross-section is $2.201$ barns
  \end{itemize} 


\emph{Another Example (2)}
     \begin{displaymath}
        f(x) = \left\{ \begin{array}{ll}
                         0.9087 & \textrm{if $x = 1$}\\                       
                         0.0454 & \textrm{if $x = 2$}\\
                         0.0005 & \textrm{if $x = 3$}\\
                         0.0454 & \textrm{if $x = 4$}\\
                         0 & \textrm{otherwise}
                       \end{array} \right.
     \end{displaymath}
     \begin{displaymath}
        F(x) = \left\{ \begin{array}{ll}
                         0.0    & \textrm{if $x \le 0$}\\                       
                         0.9087 & \textrm{if $0 < x \le 1$}\\
                         0.9541 & \textrm{if $1 < x \le 2$}\\
                         0.9546 & \textrm{if $2 < x \le 3$}\\
                         1.0 & \textrm{if $x > 3$}
                       \end{array} \right.
     \end{displaymath}




\emph{Properties of all cdf's}
   $F(x)$ is non-decreasing in $x$, i.e., $x_{1} < x_{2}$ implies that $F(x_{1}) \le F(x_{2})$
   \begin{equation*}
    \lim_{x \to \infty} F(x) = 1
   \end{equation*}
   \begin{equation*}
    \lim_{x \to -\infty} F(x) = 0
   \end{equation*}  


\section{Statistics}

\subsection{Mean and Variance of a Distribution}

\emph{Mean}
  The mean or expected value of a random variable $X$ is
  \begin{equation*}
    \mu \equiv E[X] \equiv \int_{\mathbb{R}}xf(x)dx \; \textrm{if $X$ is continuous}
  \end{equation*}
  \begin{equation*}
    \mu \equiv E[X] \equiv \sum_{x}xf(x) \; \textrm{if $X$ is discrete}
  \end{equation*}
  \begin{itemize}
   \item The mean gives an indication of the central tendency of $X$.
  \end{itemize}


\emph{Variance}
  The variance of a random variable $X$ is the second central moment.
  \begin{equation*}
    \sigma^{2} \equiv Var(X) \equiv E[(X-\mu)^{2}] = \int_{\mathbb{R}}(x-\mu)^{2}f(x)dx
  \end{equation*}
  \begin{itemize}
   \item The variance mean gives an indication of spread or dispersion.
   \item The standard deviation of $X$ is $\sigma \equiv \sqrt{Var(X)}$
  \end{itemize}


\emph{More Generally}
  The $k^{th}$ moment of $X$ is
  \begin{equation*}
    E[X^k] = \int_{\mathbb{R}}x^k f(x) dx
  \end{equation*}
  The $k^{th}$ central moment of $X$ is
  \begin{equation*}
    E[(X-\mu)^k] = \int_{\mathbb{R}}(x-\mu)^k f(x) dx
  \end{equation*}
  \begin{itemize}
   \item The third and fourth central moment are used to define skewness (measure of asymmetry) and kurtosis (measure of peakedness).
  \end{itemize}


\emph{Simpler way to get Variance}
 Theorem: $Var(X) = E[X^2] - (E[X])^2$\\
 Proof:
 % show proof in class


\emph{Scaling}
 What happens to the mean and variance if the random variable is scaled?
  \begin{itemize}
     \item Mean: $E[aX+b] = aE[X]+b$
     \item Variance: $Var(aX+b) = a^2 Var(X)$
     % Show proofs
  \end{itemize}


\emph{More Definitions}
 If $X_{1}$ and $X_{2}$ are independent random variables, and $S$ and $T$ are sets of numbers, then
  \begin{equation*}
    P(X_1 \in S \; \textrm{and} \; X_2 \in T) = P(X_1 \in S)P(X_2 \in T)
  \end{equation*}
 If $X_{1}$ and $X_{2}$ are independent random variables with variances $\sigma_{X_1}^2$ and $\sigma_{X_2}^2$, then the variance of the sum $X_1+X_2$ is
  \begin{equation*}
    \sigma_{X_1+X_2}^2 = \sigma_{X_1}^2 + \sigma_{X_2}^2 
  \end{equation*}


\emph{For our purposes}
 Each simulated neutron is considered to be an independent random variable
  \begin{itemize}
   \item Equivalent to assuming that neutron transport is linear
   \begin{itemize}
   \item Neutrons do not alter their environment (i.e. no material depletion after single event, no strutural changes after collisions, no temperature changes after single event)
   \item Neutrons do not interact with each other
   \end{itemize}
   \item What if this wasn't true?
   %Discuss non-linearity issue (w/ T feedback, depletion, compare with charged particles)
  \end{itemize}


\subsection{Mean and Variance of a Sample}

\emph{Sample Mean and Sample Variance}
 If $X_1,\dots,X_n$ is an independent random sample from a population with mean $\mu$ and variance $\sigma^2$, then the sample mean $\overline{X}$ is also a random variable with
 \begin{equation*}
  \mu_{\overline{X}} = \mu
 \end{equation*}
 \begin{equation*}
  \sigma_{\overline{X}}^2=\sigma^2/n
 \end{equation*}
 \begin{itemize}
  \item This is a very important results, because it indicates the relation between variance and the number of indepenent samples.
 \end{itemize}
 % Show proof



% -----------------------------------------------------------------------------
\section{Conclusions}
\subsection{Recap}

\begin{itemize}
   \item ALWAYS verify that you truly have a pdf/pmf before starting to sample.
   \item Average value is never enough in Monte Carlo simulations, you must ALWAYS present a measure of related statistical error.
   \item The population variance is proportional to $1/N$ which can indicate how many more independent realizations are necessary to attain a desired level of error.
  \end{itemize}

References
  \begin{itemize}
   \item F. Brown's Lecture Notes
   \item 1987 Issue of \emph{ Los Alamos Science}
  \end{itemize}
