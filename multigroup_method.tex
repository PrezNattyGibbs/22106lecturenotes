\chapter{Multigroup Method}
\label{lec:multigroup_method}

In this lecture, the energy dependence of the particle flux is 
treated using the multigroup method.  With knowledge of the 
energy spectrum (as computed, e.g., using the approach of 
Lecture~\ref{lec:neutron_slowing_down}), cross sections can be 
averaged over energy intervals in which various reaction 
rates are preserved.  Not just any averaging will do, however; rather, we
need to employ flux-weighted averaging.

To be continued.

\section*{Further Reading}

To be continued.

\begin{exercises}

  %----------------------------------------------------------------------------%
  \item \textbf{Generation of Multigroup Constants}.
    For the same three cases studied in the last lecture, 
    compute multigroup, microscopic, capture 
    cross sections for U-238 over the 
    ranges $E\in[1, 12]$, $E\in[12, 28]$, and $E\in [28, 50]$, all in 
    \electronvolt.  Note that these groups each contain one of the 
    first three resonances of U-238.  
    Use the numerical, NR, and WR spectra.  Taking 
    the numerical solution to be the reference, compute the relative 
    error of the NR and WR results for each of the three groups and
    hydrogen concentrations.
    
  %----------------------------------------------------------------------------%
  \item \textbf{Self-Shielding Factors}.
    For the same three energy groups, compute the (Bondarenko) self-shielding 
    factors using the three different spectra.
    
  
\end{exercises}
