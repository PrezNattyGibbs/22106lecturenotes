\chapter{Numerical Diffusion}
\label{lec:numerical_diffusion}

In \LECTURE{pn_method_and_diffusion}, the diffusion approximation 
was formally derived.  In this lecture,  we present an approach 
for solving the diffusion equations numerically.  Diffusion theory 
represents an important method by itself.  Moreover, it finds widespread
use in the acceleration of higher-order transport methods, including 
the nonlinear diffusion acceleration of \LECTURE{nonlinear_acceleration} and
the linear preconditioners of \LECTURE{linear_acceleration}.

\section*{Mesh-Centered Diffusion Approximation}

The diffusion approximation can also be coupled with the 
multigroup treatment in multiple spatial dimensions.  Here,
we limit the discussion to the multidimensional, one-group
diffusion equation.  A full multigroup treatment is left as 
an exercise.

Consider the one-group diffusion equation
\begin{equation}
  -\nabla D(\vec{r}) \nabla \phi(\vec{r}) + \Sigma_r(\vec{r})\phi(\vec{r})
    = S(\vec{r}) \, ,
\label{eq:1gdiff}
\end{equation}
where $\Sigma_r$ is the ``removal'' cross section.  In the one-group
approximation, $\Sigma_r = \Sigma_a$.
To solve \EQ{eq:1gdiff} numerically, we use 
a mesh-centered, finite-volume approach, 
in which cell materials and sources are taken to be 
constant \cite{hebert2009arp}. 
By integrating \EQ{eq:1gdiff} over the volume $V_{ijk}$, one obtains
\begin{equation}
\begin{split}
 \int^{x_{i+1/2}}_{x_{i-1/2}} dx 
   \int^{y_{j+1/2}}_{y_{j-1/2}} dy 
     & \int^{z_{k+1/2}}_{z_{k-1/2}} dz 
    \Bigg \{ -\nabla D(\vec{r}) \nabla \phi(\vec{r})  
              + \Sigma_r(x,y,z) \phi(x,y,z) \Bigg \} \\
       &= \int^{x_{i+1/2}}_{x_{i-1/2}} dx 
            \int^{y_{j+1/2}}_{y_{j-1/2}} dy 
              \int^{z_{k+1/2}}_{z_{k-1/2}} dz \, S(x,y,z)  \, ,
\end{split}
\end{equation}
or
\begin{equation}
\begin{split}
 -D_{ijk} \Bigg [ \Delta_{y_j}  \Delta_{z_k} 
                   \Big (\phi_x(x_{i+1/2},y_j,z_k) &- \phi_x(x_{i-1/2},y_j,z_k) \Big ) \\
                 + \Delta_{x_i} \Delta_{z_k} 
                   \Big (\phi_y(x_i,y_{j+1/2},z_k) &- \phi_y(x_i,y_{j-1/2},z_k) \Big ) \\
                 + \Delta_{x_i}  \Delta_{y_j} 
                   \Big (\phi_z(x_i, y_j, z_{k+1/2}) &- \phi_z(x_i, y_j, z_{k-1/2}) \Big ) 
          \Bigg ] \\
     + \Delta_{x_i} \Delta_{y_j} \Delta_{z_k}  \Sigma_{r,ijk} \phi_{ijk} 
      &= \Delta_{x_i} \Delta_{y_j} \Delta_{z_k} S_{ijz} \, ,
\end{split}
\label{eq:integrateddiffeq}
\end{equation}
where the cell-centered flux (equal to the average flux) is defined as
\begin{equation}
 \phi_{ijk} \equiv 
   \frac{1}{\Delta_{x_i}}\frac{1}{\Delta_{y_j}}\frac{1}{\Delta_{z_k}} 
     \int^{x_{i+1/2}}_{x_{i-1/2}} dx 
     \int^{y_{j+1/2}}_{y_{j-1/2}} dy 
     \int^{z_{k+1/2}}_{z_{k-1/2}} dz \, \phi(x,y,z) \, ,
\end{equation}
the cell-average source is
\begin{equation}
 S_{ijk} \equiv 
   \frac{1}{\Delta_{x_i}}\frac{1}{\Delta_{y_j}}\frac{1}{\Delta_{z_k}} 
     \int^{x_{i+1/2}}_{x_{i-1/2}} dx 
     \int^{y_{j+1/2}}_{y_{j-1/2}} dy 
     \int^{z_{k+1/2}}_{z_{k-1/2}} dz \, S(x,y,z) \, ,
\end{equation}
and, for example, $\phi_x(x_{i+1/2},y_j,z_k)$ is the derivative of $\phi$ with 
respect to $x$, averaged over $y$ and $z$, and evaluated at $x=x_{i+1/2}$.

To evaluate the partial derivatives $\phi_x$, $\phi_y$, and $\phi_z$ in 
\EQ{eq:integrateddiffeq}, Taylor-series expansions are 
employed. For the
$x$-directed terms at the left (or west) boundary, let
\begin{equation}
  \phi(x_{i-1},y_j,z_k)  \approx 
      \phi(x_{i-1/2},y_j,z_k) - 
      \frac{\Delta x_{i-1}}{2} \phi^+_x (x_{i-1/2},y_j,z_k) 
 \label{eq:leftXexpA} 
\end{equation}
and
\begin{equation}
  \phi(x_{i},y_j,z_k)    \approx 
      \phi(x_{i-1/2},y_j,z_k) + 
      \frac{\Delta x_{i}}{2} \phi^-_x (x_{i-1/2},y_j,z_k) \, ,
 \label{eq:leftXexpB}
\end{equation}
with similar expressions for the right $x$ boundary and the 
$y$- and $z$-directed terms.  The $+$ and $-$ superscripts
on the partial derivatives indicate forward and backward extrapolation
from the cell midpoint,
respectively.
By continuity of net current, we must have
\begin{equation}
   D_{i-1,jk}\phi^+_x(x_{i-1/2},y_j,z_k) 
     = D_{ijz}\phi^-_x(x_{i-1/2},y_j,z_k) \, .
 \label{eq:contcurr}
\end{equation}
Multiplication of \EQ{eq:leftXexpA} by 
$D_{i-1,jk}/\Delta x_{i-1}$ and
\EQ{eq:leftXexpB} by $D_{ijk}/\Delta x_{i}$,
adding the results, and rearranging leads to
\begin{equation}
 \phi_{i-1/2,jk} = 
   \frac{D_{i-1,jk} \phi_{i-1,jk} \Delta_{x_i} + 
             D_{ijk} \phi_{ijk} \Delta x_{i-1}}
        {D_{i-1,jk} \Delta_{x_i} + D_{ijk} \Delta x_{i-1}} \, .
\end{equation}
Substituting this into the \EQ{eq:leftXexpB} gives
\begin{equation}
\begin{split}
 \phi_x(x_{i-1/2},y_j,z_k) &= 
  \frac{2}{\Delta_{x_i}} 
    \Bigg (\phi_{ijk} - 
           \frac{D_{i-1,jk} \phi_{i-1,jk} \Delta_{x_i} + 
                     D_{ijk} \phi_{ijk} \Delta x_{i-1}}
                {D_{i-1,jk} \Delta_{x_i} + D_{ijk} \Delta x_{i-1}} 
    \Bigg ) \\
  &= \frac{2}{\Delta_{x_i}} 
     \Bigg ( \frac{D_{i-1,jk} \phi_{ijk} \Delta_{x_i} + 
                       D_{ijk} \phi_{i-1,j,k} \Delta x_{i-1}}
                  {D_{i-1,jk}  \Delta_{x_i} + 
                       D_{ijk} \Delta x_{i-1}} \Bigg ) \\  
\end{split}
\end{equation}
or
\begin{equation}
  \phi_x(x_{i-1/2},y_j,z_k) = 
    2D_{i-1,jk} \Bigg ( \frac{\phi_{ijk} - \phi_{i-1,jk}} 
                             {\Delta_{x_i} D_{i-1,jk} + 
                              \Delta x_{i-1} D_{ijk}} 
                \Bigg ) \, ,  
\label{eq:phideriv1}
\end{equation}
Similarly, one finds
\begin{equation}
  \phi_x(x_{i+1/2},y_j,z_k) = 
    2D_{i+1,jk} \Bigg ( \frac{\phi_{i+1,jk}-\phi_{ijk}} 
                             {\Delta_{x_i} D_{i+1,jk} + 
                              \Delta x_{i+1} D_{ijk}} 
                \Bigg ) \,  .
\label{eq:phideriv2}
\end{equation}
These equations and the equivalents for $y$ and $z$ 
are substituted into
\EQ{eq:integrateddiffeq} to obtain a set of \emph{internal equations}.


\subsubsection{Internal Equation}
Substitution of Eqs.~(\ref{eq:phideriv1})~and~(\ref{eq:phideriv2})
into  \EQ{eq:integrateddiffeq} leads to
\begin{equation}
\begin{split}
  -D_{ijk} 
   \Bigg \{  
    \Delta_{y_j} \Delta z_j 
    \Bigg [ 
       2D_{i+1,jk} \Bigg (& 
                  \frac{ \phi_{i+1,j}-\phi_{i,j} } 
                       { \Delta_{x_i} D_{i+1,jk} + \Delta x_{i+1} D_{ijk} } 
                  \Bigg ) +  \\
       2D_{i-1,jk} \Bigg (& 
                  \frac{\phi_{i-1,j} - \phi_{i,j}} 
                       { \Delta_{x_i} D_{i-1,jk} + \Delta x_{i-1} D_{ijk} } 
                 \Bigg ) 
    \Bigg ] \ldots + 
   \Bigg \} + \\
   & \Delta_{x_i} \Delta_{y_j} \Delta_{z_k} \Sigma_{r,ijk} \phi_{ijk} 
   = \Delta_{x_i} \Delta_{y_j} \Delta_{z_k} S_{ijk} \, .
\end{split}
\label{eq:inteqs}
\end{equation}  
\EQUATION{eq:inteqs} represents the balance of neutrons in the
cell $(i,j,k)$.

We can further simplify the notation.  Let us define a coupling coefficient
\begin{equation}
 \tilde{D}_{i+1/2,jk} \equiv 
   \frac{2D_{i+1,jk}D_{ijk}}
        {\Delta_{x_i} D_{i+1,jk} + \Delta_{x_{i+1}} D_{ijk} } \, ,
\end{equation}
with similar coefficients for each direction.  Then 
we can rewrite Eq. \ref{eq:inteqs} as
\begin{equation}
 \begin{split}
     \frac{ \tilde{D}_{i+1/2,j,k}}{\Delta_{x_i}} 
         \Big (\phi_{ijk} - \phi_{i+1,j,k}  \Big ) 
  & +\frac{ \tilde{D}_{i-1/2,j,k}}{\Delta_{x_i}} 
         \Big (\phi_{ijk} - \phi_{i-1,j,k}  \Big ) \\
    +\frac{ \tilde{D}_{i,j+1/2,k}}{\Delta_{y_j}} 
         \Big (\phi_{ijk} - \phi_{i,j+1,k} \Big ) 
  & +\frac{ \tilde{D}_{i,j-1/2,k}}{\Delta_{y_j}} 
         \Big (\phi_{ijk} - \phi_{i,j-1,k} \Big ) \\
    +\frac{ \tilde{D}_{i,j,k+1/2}}{\Delta_{z_k}} 
         \Big (\phi_{ijk} - \phi_{i,j,k+1} \Big ) 
  & +\frac{ \tilde{D}_{i,j,k-1/2}}{\Delta_{z_k}}
         \Big (\phi_{ijk} - \phi_{i,j,k-1} \Big ) \\
  & + \Sigma_{r,ijk} \phi_{ijk} =  S_{ijk} \, .
 \end{split}
 \label{eq:inteqssimp}
\end{equation}
Note that each term on the left with a coupling coefficient represents
the net leakage from a surface divided by the area of that surface.

\subsubsection{Boundary Equations}

At the boundaries, we employ the albedo condition \cite{hebert2009arp}
\begin{equation}
 \frac{1}{2} D(x,y,z) \nabla \phi(x,y,z) \cdot \hat{n}(x,y,z) + 
 \frac{1}{4} \frac{1-\alpha(x,y,z)}{1+\alpha(x,y,z)}\phi(x,y,z) = 0 \, ,
\label{eq:albedo}
\end{equation}
where $\hat{n}$ is the outward normal, $\alpha$ describes the albedo 
condition ($\alpha = 0$ for vacuum, and $\alpha = 1$
for reflection).  

In three dimensions, a mesh cell has six surfaces, some of which 
may be part of a global surface.  As an example,
we consider the west global boundary:
\begin{equation}
 \text{west boundary}:  \,\,\,  
    -\frac{1}{2}D_{1jk} \phi_x(x_{1/2},y_j,z_k) + 
   \frac{1}{4}\frac{1-\alpha}{1+\alpha}\phi_{1/2,jk} = 0 \, .
\end{equation}
Because this expression contains $\phi$ at the edge, we again need to 
employ Taylor expansions.  For the west boundary, note that
\begin{equation}
 \phi_{1jk} \approx \phi_{1/2,jk} + \frac{\Delta_{x_i}}{2} \phi_x(x_{1/2},y_j,z_k) \, ,
\end{equation}
and we rearrange to get
\begin{equation}
 \phi_{1/2,jk} = \phi_{1jk} - \frac{\Delta_{x_i}}{2} \phi_x(x_{1/2},y_j,z_k) \, .
\end{equation}
Placing this into the albedo condition yields
\begin{equation}
\begin{split}
 0 =  -\frac{1}{2}D_{1jk} \phi_x(x_{1/2},y_j,z_k)  
  + \frac{1}{4}\frac{1-\alpha}{1+\alpha} 
    \Bigg ( \phi_{1jk} - \frac{\Delta_{x_i}}{2} \phi_x(x_{1/2},y_j,z_k) \Bigg ) \, ,
\end{split}
\end{equation}
and solving for $\phi_x(x_{1/2},y_j,z_k)$ gives
\begin{equation}
 \phi_x(x_{  1/2},y_j,z_k) = \frac{2(1-\alpha)\phi_{1jk}}
                                  {4(1+\alpha)D_{1jk} + \Delta_{x_1}(1-\alpha)} \, .
\end{equation}
For the east, we similarly find
\begin{equation}
 \phi_x(x_{I+1/2},y_j,z_k) =-\frac{2(1-\alpha)\phi_{Ijk} }
                                  {4(1+\alpha)D_{Ijk} + \Delta_{x_I}(1-\alpha)} \, ,
\end{equation}
and likewise for the other surfaces. Each of these  is placed into the proper 
partial derivative of \EQ{eq:integrateddiffeq}.  For example, the leakage
contribution on the left hand side of \EQ{eq:inteqssimp} due to leakage from 
the west boundary is transformed as 
follows\footnote{Ignore the fact that $\phi_{0jk}$ doesn't exist!}: 
\begin{equation}
  \frac{ \tilde{D}_{1/2,jk}}{\Delta_{x_1}}  \Big (\phi_{0jk} - \phi_{1jk}  \Big )  \rightarrow 
     \frac{2D_{1jk}(1-\alpha) \phi_{1jk}}
      {(4(1+\alpha)D_{1jk} + \Delta_{x_1}(1-\alpha))\Delta_{x_1}} \, .
\label{eq:west_leakage}      
\end{equation}


\subsubsection{Constructing the Diffusion Matrix}

The equations for cell balance derived heretofore can be combined into 
a linear system for the cell fluxes.  Generically, we can express this 
system as
\begin{equation}
 \mathbf{L}\vec{\phi} = \vec{S} \, ,
\end{equation}
where $\mathbf{L}$ is the  diffusion loss matrix. 
A straightforward implementation of the one-group diffusion loss matrix 
is presented in Listing~\ref{list:lossmatrix}.  Note that several 
parameters and helper functions require definition.  Although the 
code is set up for three dimensions, it can also be used for 1-D 
and 2-D problems by proper selection of boundary conditions.

\lstset{language=Python,caption=Sample code for construction of one-group loss matrix., label=list:lossmatrix, morecomment=[l]{\%}}
\begin{lstlisting}
L   = zeros((N, N))             # N x N loss matrix
remdims = [[1,2], [0,2], [0,1]] # For a given dim, provides remaining dims
# Loop over all cells
for n in range(0, N) :
  ijk = cell_to_ijk(n)          # Convert cardinal index to (i, j, k)
  w = [dx[ijk[0]], dy[ijk[1]], dz[ijk[2]]]
  # Loop over dimensions
  for dim in range(0, 3) :
    bound = [0, Nxyz[dim]]      # Minimum and maximum cell index in this dim
    # Loop over directions, + and -
    for dir in range(0, 2) :
      if ijk[dim] == bound[dir] :
        # Global boundary
        side = 3*dim + dir      # index of this side (0-5)
        B = beta[side]          # albedo for this side
        L[n][n] = L[n][n] + 2*D[n]*(1-B)/((4*(1+B)*D[n]+w[dir]*(1-B))*w[dir])
      else :
        # Coupling to neighbor at cell m
        ijk_m = ijk
        ijk_m[dim] = ijk_m[dim] + (-1)**dir
        m = cell_from_ijk(ijk_m)
        w_m = [dx[ijk_m[0]], dy[ijk_m[1]], dz[ijk_m[2]]]
        D_tilde = 2*D[n]*D[m]/((w[dir]*D[m]+w_m[dir]*D[n])*w[dir])
        L[n][m] = -D_tilde / w[dir]
        L[n][n] = L[n][n] + D_tilde / w[dir]
  # Add the removal cross-section component
  L[n][n] = L[n][n] + Sigma_r[n]
\end{lstlisting}


\begin{exercises}
 \item \textbf{Boundary Source}.  Suppose a partial current 
   $J^{\mathrm{west}-}_{jk}$ were incident on the west boundary 
   of cell $(1, j, k)$.  Show that the corresponding current 
   balance contribution (i.e., \EQ{eq:west_leakage}) is unchanged 
   but that the following source term arises:
  \begin{equation}
  Q^{\mathrm{boundary}}_{1jk} = \frac{8 D_{1jk} (1+\alpha) J^{\mathrm{west}-}_{jk}}
                                    {(4(1+\alpha)D_{1jk} + \Delta_{x_1}(1-\alpha))\Delta_{x_1}} 
  \end{equation}
 
  \item \textbf{One-Group Diffusion}.  For $\Sigma_t = 1.0$ and 
  $\Sigma_s = 0.0$, compute the flux.
  
  \item \textbf{Multigroup Diffusion}. 
    Although a formal derivation of the multigroup diffusion approximation
    requires assumptions beyond the ones
    made in \LECTURE{pn_method_and_diffusion},
    the resulting set of multigroup diffusion equations is intuitive:
    \begin{equation}
    \begin{split}
      -\nabla D_g(\vec{r}) \nabla \phi_g(\vec{r}) &+ 
        \Sigma_{rg}(\vec{r})  \psi_g(\vec{r})  = 
        \sum_{\substack{g'=1 \\ g'\neq g}}^{G} 
          \Sigma_{sg \gets g'}(\vec{r}) \phi_{g'}(\vec{r})  \\
        &+  \frac{\chi_g(\vec{r})}{ k}  
              \sum_{g'=1}^{G} 
                \nu\Sigma_{fg'}(\vec{r}) 
                  \phi_{g'}(\vec{r})
        +  S_g(\vec{r})  \, .
    \end{split}
    \label{eq:multigroupneutrondiffusionequation}
    \end{equation}
    Extend the routine proposed in Listing~\ref{list:lossmatrix} to handle 
    multigroup problems.
  
  
\end{exercises}
