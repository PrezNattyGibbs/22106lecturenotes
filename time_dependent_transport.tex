\chapter{Time-Dependent Transport}
\label{lec:time_dependent_transport}

This lecture investigates approaches for solving the time-dependent neutron 
transport equation using the discrete ordinates method.  Several 
important historical schemes are presented. Additionally, a generalized approach for 
solving the equations via generic solvers is presented, and several options
and challenges are identified.  Throughout, delayed neutron precursors 
are explicitly included.

\section{The Time-Dependent Transport Equation}

We begin with the time-dependent, multigroup transport equation (TDTE),
\begin{equation}
\begin{split}
\overbrace{ \frac{1}{v_g} \frac{\partial \psi_g}{\partial t} }^{\text{time rate of change}} &=  
    - (\overbrace{ \hat{\Omega} \cdot \nabla \psi_g(\vec{r},\Omega,t) + 
                   \Sigma_{tg}(\vec{r},t)  \psi_g(\vec{r},\Omega,t) }^
                 {\text{rate of destruction}} ) \\ 
    &+ \overbrace{\frac{1}{4\pi} \int_{4\pi}d\Omega' \, \sum_{g'=1}^{G} 
                  \Sigma_{sg \gets g'}(\vec{r},\Omega' \cdot \Omega,t) \psi_{g'}(\vec{r},\Omega',t)}^
                 {\text{production rate from in-scatter}} \\
    &+ \overbrace{\frac{1}{4\pi k_{\text{crit}} }  \int_{4\pi}d\Omega' \, 
                    \Bigg ( 
                       \chi_{pg}(\vec{r})(1-\beta(\vec{r})) \sum_{g'=1}^{G} 
                       \nu\Sigma_{fg'}(\vec{r},t)  \psi_{g'}(\vec{r},\Omega',t) 
                    \Bigg )}^
                 {\text{production rate from prompt fission}} \\
    &+ \overbrace{\sum_i \lambda_i C_i(\vec{r}, t) \chi_{ig}(\vec{r})}^
                 {\text{production rate from delayed fission}} \\
    &+ \overbrace{q_g(\vec{r}, \hat{\Omega}, t)}^
                 {\text{production rate from external sources}}\, ,  \,\,\, g = 1 \ldots G
\end{split}
\label{eq:tdte}
\end{equation}
with a corresponding set of precursor equations,
\begin{equation}
 \frac{\partial C_i}{\partial t} =  
   \frac{\beta_i(\vec{r})}{k_{\text{crit}}} \int_{4\pi}d\Omega' \,  
     \sum_{g'=1}^{G} \nu\Sigma_{fg'}(\vec{r},t)  
       \psi_{g'}(\vec{r},\Omega',t) 
  - \lambda_i C_i(\vec{r},t) \, , \,\,\, i = 1\ldots N \, .
\label{eq:prec}
\end{equation}
While the notation is essentially standard, it does imply 
isotropic scattering in the LAB system.  Furthermore, it
assumes that 
a steady-state calculation has been performed for the critical 
$k$-eigenvalue and
use of modern delayed neutron data such that a single set of 
decay constants $\lambda_i$ is applicable to all materials.

Before proceeding, we adopt the operator notation developed 
in \LECTURE{operator} to simplify
somewhat the exposition to follow.
For the time-dependent multigroup problem, this notation generalizes to 
\begin{equation}
  \frac{1}{v_g} \frac{\partial \psi_g}{\partial t}
   = -\mathcal{L}_g\psi_g
     + \mathcal{M} \sum_{g'=1}^G 
         \left (
           \mathcal{S}_{gg'} + 
           \mathcal{X}_{pg} \mathcal{F}_{g'} 
         \right )\phi_{g'}
     +  \mathcal{M} \sum_{i=1}^{I} \lambda_i \mathcal{X}_{dig} C_i \, ,
\label{eq:tdteop}
\end{equation}
and
\begin{equation}
 \frac{\partial C_i}{\partial t} =  
  \beta_i \sum_{g=1}^G \mathcal{F}_{g} \phi_{g} - \lambda_i C_i \, ,
\label{eq:precop}
\end{equation}
where $\mathcal{F}$ has absorbed the eigenvalue, and
$\mathcal{X}_{pg} $ has absorbed the $1-\beta$ term.

\section{Time-Integration Schemes}

There appear to exist two broad categories of time integration 
approaches applicable to solving the TDTE.
The first and historically dominant approach discretizes
the equations in time in some (usually
simple) manner, and introduces ``synthetic'' materials and
sources to convert the time step into a fixed source problem.
This approach is attractive because it can make use of 
an existing fixed source code without much additional
code.  Moreover, by consistently differencing the
fluxes and precursors, it is possible to eliminate the 
precursors as an unknown, thus reducing the size of 
the system.  
A drawback to this approach is that for complex
time integration schemes, the modified coefficients may 
become arbitrarily complex. Any change in a scheme would
require different coefficients.  Furthermore, coupling
to other physics would be limited to operator splitting,
possibly with an inner iteration to correct.

A second approach is to consider the TDTE as a generic
ODE for use in general time integration schemes.  This
approach is attractive because arbitrarily high order
integration schemes would be applicable.  Moreover,
it would be comparitively easier to couple other 
physics, yielding one consistent set of ODE's.  
A significant potential downside to this approach is that
the operators used to define the ODE may be more 
difficult to apply and may require significant new
code.  
Interestingly, and perhaps telling of these challenges,
it appears no production codes use this latter approach.

\section{A Survey of Historical Time Integration Schemes}

At least three variants of the first approach are found in
production codes of the last several decades.  Much of 
the discussion to follow is similar to the review given
by Dishaw \cite{dishaw2007tdd}, though we have made 
corrections and our own observations where
needed.  In all cases, where applicable, the historical 
schemese make use of synthetic sources and materials in order 
to convert the time-dependent problem into 
a series of fixed-source problems.

\subsection{TIMEX-1972}

In the first incarnation of the Los Alamos TIMEX code, the time 
discretization is defined using a semi-implicit,
first order discretization defined by
\begin{equation}
  \frac{1}{\Delta_t v_g} \left ( \psi^{n+1}_g - \psi^{n}_g \right )
     + \mathcal{L}\psi^{n+1}_g = 
       \mathcal{M} \sum_{g'=1}^G 
         \left (
           \mathcal{S}_{gg'} + 
           \mathcal{X}_{pg} \mathcal{F}_{g'} 
         \right )\phi^{n}_{g'}
     +  \mathcal{M} \sum_{i=1}^{I} \lambda_i \mathcal{X}_{dig} C^{n}_i \, ,
\label{eq:timex1972}
\end{equation}
where $n$ indexes the time step.  Note that the latest time
step $n+1$ occurs only on the left hand side; hence, the 
implicit component of this step is relatively small.  Note,
original documentation for this method was not found; hence, 
we rely on Dishaw's review \cite{dishaw2007tdd}.

Substituting
\begin{equation}
 \tilde{\Sigma}_{tg} \equiv \Sigma_{tg} + \frac{1}{\Delta_t v_g} \, ,
\end{equation}
with a corresponding change to $\tilde{\mathcal{L}}$ and defining
\begin{equation}
 \tilde{q}^{n+1} \equiv
     \mathcal{M} \sum_{g'=1}^G 
         \left (
           \mathcal{S}_{gg'} + 
           \mathcal{X}_{pg} \mathcal{F}_{g'} 
         \right )\phi^{n}_{g'}
     +  \mathcal{M} \sum_{i=1}^{I} \lambda_i \mathcal{X}_{dig} C^{n}_i
     - \frac{1}{\Delta_t v_g} \psi^{n}_g \, ,
\end{equation}
yields the equation
\begin{equation}
  \tilde{\mathcal{L}}_g\psi^{n+1}_g = \tilde{q}^{n+1}_g \, ,
\end{equation}
which is a one group fixed source 
problem with pure absorption.  The precursors are 
updated based on the new flux.

This method has two chief downsides.  The first is its inherent
instability for large time steps being only slightly implicit.  
The second is its inherent inaccuracy caused by lagging all
scattering and fission sources.  What this implies is that the
scattering and fission sources are never converged at a time 
step, a potentially huge source of error.
The method was attractive because of its efficiency: only a 
single transport sweep is required per step.

\subsection{TIMEX-1976}

A second incarnation of TIMEX \cite{hill1976tim} 
improved its time discretization,
using a modified semi-implicit scheme defined by
\begin{equation}
\begin{split}
  \frac{1}{\Delta_t v_g} \left ( \psi^{n+1}_g - \psi^{n}_g \right )
     + \mathcal{L}\psi^{n+1}_g &= 
       \mathcal{M} \sum_{g'=1}^{g-1} 
         \left (
           \mathcal{S}_{gg'} + 
           \mathcal{X}_{pg} \mathcal{F}_{g'} 
         \right )\phi^{n+1}_{g'}  \\
    &+ \mathcal{M} \sum_{g'=g}^{G} 
         \left (
           \mathcal{S}_{gg'} + 
           \mathcal{X}_{pg} \mathcal{F}_{g'} 
         \right )\phi^{n}_{g'}
     +  \mathcal{M} \sum_{i=1}^{I} \lambda_i \mathcal{X}_{dig} C^{n}_i \, .
\end{split}
\label{timex1976}
\end{equation}
 
This updated scheme introduces something similar to 
Gauss-Seidel iteration for defining the scattering and
fission sources, though it converges neither.  Consequently, this
scheme is still subject to gross inaccuracy.  
As done for TIMEX 1972, we can define synthetic materials
and sources.  Using $\tilde{\mathcal{L}}$ as above and defining
\begin{equation}
 \tilde{q}_g =        \mathcal{M} \sum_{g'=1}^{g-1} 
         \left (
           \mathcal{S}_{gg'} + 
           \mathcal{X}_{pg} \mathcal{F}_{g'} 
         \right )\phi^{n+1}_{g'}  
     + \mathcal{M} \sum_{g'=g}^{G} 
         \left (
           \mathcal{S}_{gg'} + 
           \mathcal{X}_{pg} \mathcal{F}_{g'} 
         \right )\phi^{n}_{g'}
     +  \mathcal{M} \sum_{i=1}^{I} \lambda_i \mathcal{X}_{dig} C^{n}_i \, ,
\end{equation}
we have the equation
\begin{equation}
 \tilde{\mathcal{L}}\psi_g = \tilde{q}_g \, ,
\end{equation}
which, as expected, has the same form as above.

\subsection{PARTISN}

A successor to TIMEX is the Los Alamos code 
PARTISN \cite{alcouffe2005par},
which remains a standard production code for many
institutions today.
The time discretization in PARTISN  is equivalent 
to the the implicit midpoint method \cite{alcouffe1998par}.  Consider the ODE
\begin{equation}
 \frac{d y(t)}{dt} = f(t, y(t)) \, .
\end{equation}
The implicit midpoint method defines the step
\begin{equation}
 y^{n+1} = y^n + \Delta_t f \left (t^n + \frac{\Delta_t}{2},
                                   \frac{1}{2} \left (y^{n+1} + y^{n} \right)  \right ) \, .
\end{equation}
This can implemented as a sequence of two steps.  Defining
the midpoint value
\begin{equation}
 y^{n+\frac{1}{2}} =  \frac{1}{2} \left (y^{n+1} + y^{n} \right) \, ,
\end{equation}
we perform a backward Euler step,
\begin{equation}
 y^{n+\frac{1}{2}} = y^n + \frac{\Delta_t}{2} 
    f \left (t+\frac{\Delta_t}{2}, y^{n+\frac{1}{2}} \right ) \, .
\end{equation}
Using the computed $y^{n+\frac{1}{2}}$ and the midpoint definition, we 
then extrapolate to $t^{n+1}$ 
using
\begin{equation}
 y^{n+1} =  2y^{n+\frac{1}{2}} - y^n \, .
\end{equation}
It can be shown this is a second order accurate method.  

We can apply this method to the TDTE.  The fully implicit
backward Euler half-step is defined via
\begin{equation}
  \frac{2}{\Delta_t v_g} \left ( \psi^{n+\frac{1}{2}}_g - \psi^{n}_g \right )
     + \mathcal{L}\psi^{n+\frac{1}{2}}_g = 
       \mathcal{M} \sum_{g'=1}^G 
         \left (
           \mathcal{S}_{gg'} + 
           \mathcal{X}_{pg} \mathcal{F}_{g'} 
         \right )\phi^{n+\frac{1}{2}}_{g'}
     +  \mathcal{M}\sum_{i=1}^{I} \lambda_i \mathcal{X}_{dig} C^{n+\frac{1}{2}}_i \, ,
\label{eq:partisnpsi}
\end{equation}
where the consistently-discretized precursors are
governed by
\begin{equation}
 \frac{2}{\Delta_t} \left ( C_i^{n+\frac{1}{2}} - C_i^{n} \right )
   = \beta_i \sum_{g=1}^G \mathcal{F}_{g} \phi^{n+\frac{1}{2}}_{g} - \lambda_i C^{n+\frac{1}{2}}_i \, .
\label{eq:partisnC}
\end{equation}
Taking these steps, the updated fluxes and precursors are
defined via
\begin{equation}
 \begin{split}
   \psi_g^{n+1} &= 2 \psi_g^{n+\frac{1}{2}} + \psi_g^{n} \\
   C_i^{n+1}    &= 2 C_i^{n+\frac{1}{2}} + C_i^{n} \, .
 \end{split}
\end{equation}

As done for the TIMEX methods, we can cast Eqs. \ref{eq:partisnpsi}
and \ref{eq:partisnC} as a fixed-source problem.  First, we 
solve \EQ{eq:partisnC} for the updated concentrations,
\begin{equation}
   C_i^{n+\frac{1}{2}} = 
     \frac{ \Delta'_t \beta_i }{1+\Delta'_t \lambda_i} \sum_{g=1}^G \mathcal{F}_{g} \phi^{n+\frac{1}{2}}_{g} 
     + \frac{C_i^{n}}{1+\Delta'_t \lambda_i } \, ,
\end{equation}
where $\Delta'_t = \frac{\Delta_t}{2}$.
Substituting this into \EQ{eq:partisnpsi} yields
\begin{equation}
\begin{split}
  \frac{1}{\Delta'_t v_g} \left ( \psi^{n+\frac{1}{2}}_g - \psi^{n}_g \right )
     &+ \mathcal{L}_g\psi^{n+\frac{1}{2}}_g = 
       \mathcal{M} \sum_{g'=1}^G 
         \left (
           \mathcal{S}_{gg'} + 
           \mathcal{X}_{pg} \mathcal{F}_{g'} 
         \right )\phi^{n+\frac{1}{2}}_{g'} \\
    &+  \mathcal{M}\sum_{i=1}^{I} \lambda_i \mathcal{X}_{dig} 
     \left ( 
       \frac{ \Delta'_t \beta_i }{1+\Delta'_t \lambda_i} \sum_{g'=1}^G \mathcal{F}_{g'} \phi^{n+\frac{1}{2}}_{g'} 
       + \frac{C_i^{n}}{1+\Delta'_t \lambda_i } 
     \right ) \, .
\end{split}
\end{equation}
Rearranging, we find
\begin{equation}
\begin{split}
  \left ( \frac{1}{\Delta'_t v_g} + \mathcal{L}_g \right )\psi^{n+\frac{1}{2}}_g &= 
       \mathcal{M} \sum_{g'=1}^G 
         \left (
           \mathcal{S}_{gg'} + 
           \left( \mathcal{X}_{pg} +
                  \sum_{i=1}^{I} \lambda_i \mathcal{X}_{dig} \left ( \frac{ \Delta'_t \beta_i }{1+\Delta'_t \lambda_i} \right )
           \right ) \mathcal{F}_{g'} 
         \right )\phi^{n+\frac{1}{2}}_{g'} \\
   &+ \mathcal{M} \sum_{i=1}^{I} \lambda_i \mathcal{X}_{dig} 
      \left ( 
         \frac{C_i^{n}}{1+\Delta'_t \lambda_i } 
      \right ) + \frac{\psi^{n}_g}{\Delta'_t v_g} \, ,
\end{split}
\end{equation}
Defining
\begin{equation}
 \tilde{\mathcal{L}}_g \equiv \frac{1}{\Delta'_t v_g} + \mathcal{L}_g \, ,
\end{equation}
\begin{equation}
 \tilde{\chi}_g \equiv 
     \left( \mathcal{X}_{pg} +
        \sum_{i=1}^{I} \lambda_i \mathcal{X}_{dig} 
          \left ( \frac{ \Delta'_t \beta_i }{1+\Delta'_t \lambda_i} \right )
     \right )  \, ,
\end{equation}
and
\begin{equation}
 \tilde{q}^{n+\frac{1}{2}}_g \equiv 
   \mathcal{M} \sum_{i=1}^{I} \lambda_i \mathcal{X}_{dig} 
   \left ( 
     \frac{C_i^{n}}{1+\Delta'_t \lambda_i } 
   \right ) + \frac{\psi^{n}_g}{\Delta'_t v_g} \, ,
\end{equation}
we have
\begin{equation}
  \tilde{\mathcal{L}}_g \psi^{n+\frac{1}{2}}_g =  
         \mathcal{M} \sum_{g'=1}^G 
         \left (
           \mathcal{S}_{gg'} + 
           \tilde{\mathcal{X}}_{g}\mathcal{F}_{g'} 
         \right )\phi^{n+\frac{1}{2}}_{g'} 
         + \tilde{q}^{n+\frac{1}{2}}_g \, ,
\end{equation}
which is a standard fixed source multigroup problem.

The actual implementation in PARTISN is slightly different
from that presented above.  In PARTISN, the angular flux is
defined for $t^{i-\frac{1}{2}}, t^{i+\frac{1}{2}}, \ldots$
while the scalar fluxes, and, hence, reaction rates are 
computed at the centered time.  Moreover, it does not 
appear that PARTISN includes delayed neutrons. 

Another modern code, TD-TORT, also employs a fully implicit 
approach \cite{seubert2009tda}, though the exact method was
not described in the available literature.  
However, it appears the implementation used is
limited to backward Euler and hence is only first order.  


\section{Higher-Order Synthetic Schemes}

It was mentioned above that for all but the simplest time-stepping
schemes, a time discretization of the TDTE becomes quite complex.
There does exist a class of high order methods that does allow
a rather simple discretization in terms only of synthetic sources
and materials called the \emph{backward differentiation formula} (BDF) 
methods \cite{leveque2007fdm}.
The first order BDF method is equivalent to the backward Euler scheme.
The BDF methods of orders 1 through 3 are defined by
\begin{equation}
\begin{split}
               y^{n+1} -              y^{n}                                               &= \Delta_t f(t^{n+1}, y^{n+1}) \\
  \frac{ 3}{2} y^{n+1} - \frac{ 4}{ 2}y^{n} + \frac{1}{ 2} y^{n-1}                        &= \Delta_t f(t^{n+1}, y^{n+1}) \\
  \frac{11}{6} y^{n+1} - \frac{18}{ 6}y^{n} + \frac{9}{ 6} y^{n-1} - \frac{2}{6} y^{n-2} &= \Delta_t f(t^{n+1}, y^{n+1}) \, ,
\end{split}
\end{equation}
though the methods are stable through sixth order.  These can be 
written more compactly as
\begin{equation}
   \sum_{i = 0}^{m} a_i y^{n-i+1} =  \Delta_t f(t^{n+1}, y^{n+1}) \, .
\end{equation}


As done for backward Euler, we can define synthetic 
materials and sources.  In this case, defining
\begin{equation}
 \tilde{\mathcal{L}}_g \equiv \frac{a_0}{\Delta_t v_g} + \mathcal{L}_g \, ,
\end{equation}
\begin{equation}
 \tilde{\chi}_g \equiv 
     \left( \mathcal{X}_{pg} +
        \sum_{i=1}^{I} \lambda_i \mathcal{X}_{dig} 
          \left ( \frac{ \Delta_t \beta_i }{a_{0}+\Delta_t \lambda_i} \right )
     \right )  \, ,
\end{equation}
and
\begin{equation}
 \tilde{q}^{n+1}_g \equiv 
    \mathcal{M} \sum_{i=1}^{I} \left ( \frac{ \mathcal{X}_{dig}  \lambda_i}{a_0 + \lambda_i \Delta_t} \sum^{m}_{j=1} a_m C^{n-j+1}_i \right )
    -  \sum^{m}_{j=1}  \frac{a_j \psi^{n-j+1}_g}{\Delta_tv_g} \, ,
\end{equation}
yields the same standard multigroup form
\begin{equation}
  \tilde{\mathcal{L}}_g \psi^{n+1}_g =  
         \mathcal{M} \sum_{g'=1}^G 
         \left (
           \mathcal{S}_{gg'} + 
           \tilde{\mathcal{X}}_{g}\mathcal{F}_{g'} 
         \right )\phi^{n+1}_{g'} 
         + \tilde{q}^{n+1}_g \, .
\end{equation}
Of course, using a a BDF method with order $m > 1$ requires that
more than one previous state vector is stored.  The cost associated
with this higher memory use may or may not be outweighted by the
larger time steps higher order methods allow.

It is worth noting that no mention of the BDF methods for neutron
transport can be found in the literature (though admittedly the
search was not great).  Numerical studies will be needed to 
assess how valuable these schemes are for such problems.

\section{Solution Via General Solvers}

We can cast the TDTE into a form suitable for
solution via general ODE time integration schemes.  
In particular, we transform the TDTE into 
an ODE of the form
\begin{equation}
 \frac{d u(t)}{dt} = \mathcal{A}\left (t, u(t) \right) u(t)\, .
\end{equation}
Rearranging Eqs. \ref{eq:tdteop} and \ref{eq:precop}, we have
the following system of equations

\begin{equation}
\frac{d}{dt} 
\overbrace{\left[ 
  \begin{array}{c}
    \psi_1    \\
    \vdots     \\
    \psi_G    \\
    C_1       \\
    \vdots    \\
    C_I      
  \end{array} 
\right]}^{\mathbf{u}(t)}
= 
\overbrace{
\left[ 
  \begin{array}{cccccc}
     v_1 \mathcal{A}^{\psi \psi}_{11} & \cdots  & v_1 \mathcal{A}^{\psi \psi}_{1G}  & 
     v_1 \mathcal{A}^{\psi C}_{11}    & \cdots  & v_1 \mathcal{A}^{\psi C}_{I1}     \\
%
                                      & \ddots  &                                   &                      
                                      & \ddots  &                                   \\
%
     v_G \mathcal{A}^{\psi \psi}_{G1} & \cdots  & v_G \mathcal{A}^{\psi \psi}_{GG}  & 
     v_G \mathcal{A}^{\psi C}_{1I}    & \cdots  & v_G \mathcal{A}^{\psi C}_{II}     \\
%
    \mathcal{A}^{C \psi}_{11}         & \cdots  & \mathcal{A}^{C \psi}_{11}         & 
    -\lambda_1                        &         &                                   \\
%
                                      & \ddots  &                                   & 
                                      & \ddots  &                                   \\
%
    \mathcal{A}^{C \psi}_{I1}         & \cdots  & \mathcal{A}^{C \psi}_{11}         & 
                                      &         & -\lambda_I                        \\
  \end{array} 
\right]}^{\mathcal{A}(t)}
\overbrace{
\left[ 
  \begin{array}{c}
    \psi_1    \\
    \vdots     \\
    \psi_G    \\
    C_1       \\
    \vdots    \\
    C_I   
  \end{array} 
\right]}^{\mathbf{u}(t)}
\label{eq:tdtefull}
\end{equation}
where
\begin{equation}
 \mathcal{A}^{\psi \psi}_{gg'} = -\mathcal{L}_g\delta_{gg'} + \mathcal{M} \left ( \mathcal{S}_{gg'} + \mathcal{X}_{pg} \mathcal{F}_{g'} \right ) \mathcal{D} \, ,
\end{equation}
\begin{equation}
 \mathcal{A}^{\psi C}_{ig} = \lambda_i \mathcal{M} \mathcal{X}_{dig} \, ,
\end{equation}
and
\begin{equation}
 \mathcal{A}^{C \psi}_{ig} = \beta_i \mathcal{F}_{g} \mathcal{D} \, .
\end{equation}
The size of the operator $\mathcal{A}$ is quite large.  Assuming
one spatial unknown per cell, this size is
\begin{equation}
 \text{size}(\mathcal{A}) = N_{\text{cells}} \times \left ( N_\text{groups} \times N_{\text{angles}} + N_{\text{precursors}} \right ) \, .
\end{equation}
To put this in context, suppose we use an $S_8$ approximation in angle,  
2 energy groups, 8 precursor groups, and a 1 cm mesh for a typical
two-dimensional reactor benchmark problem such as the LRA problem.  
This leads to
\begin{equation}
 \text{size}(\mathcal{A}) \approx 30000 \times \left ( 2 \times 40 + 8 \right )  \approx 3\times 10^6 \, \text{unknowns} \, .
\end{equation}
While solving for millions of unknowns is not infeasible, doing
so for several thousand time steps brings the total to 
billions, and so the problem does become quite challenging.

One drawback of the form of \EQ{eq:tdtefull} is the 
challenge in defining the action of the operator $\mathcal{A}$.
Typically, the linear systems resulting from implicit 
discretizations would be solved via Krylov methods, and so
the full matrix is not necessary.  However, work has been done
to construct this full operator for the case of no 
precursors \cite{swesty2006std}.  The full explicit 
matrix is large and sparse; typical sparsity patterns are given
in Figure \ref{fig:two_d_matx} for the $\mathcal{A}^{\psi \psi}_{gg'}$
block of $\mathcal{A}$ \cite{roberts2010dsd}.

% \begin{figure}[ht]
%   \centering
%     \subfigure[full matrix]{
%       \includegraphics[keepaspectratio, width = 3.0 in]{two_d_matx_full}
%      }
%      \subfigure[similar to (a), and zoomed]{
%        \includegraphics[keepaspectratio, width = 3.0 in]{two_d_matx_zoom}
%      }
%   \caption{2-D, 2-G, 20 $\times$ 20 mesh, S$_4$ in (a)}
%   \label{fig:two_d_matx}
% \end{figure}

An alternative to constructing the operator explicitly would be to
define only its action.  Most of the action $\mathcal{A}$ is
straightforward; for example, a term such as 
$\mathcal{M} \mathcal{S}_{gg'} \mathcal{D} \psi_{g'}$ represents a typical
source construction likely present in most codes in some form.  However,
the action $\mathcal{L}_g \psi_{g}$ is not likely found.  This is because
discrete ordinates codes solve the problem via sweeping across the domain
for all angles.  This sweep is represented by the \emph{inverse} of the 
operator $\mathcal{L}_g $.  To be more explicit, consider the 
equation
\begin{equation}
 \mathcal{L}_g \psi_g = \mathcal{M} \mathcal{S}_{gg} \mathcal{D} \psi_g + q_g \, .
\end{equation}
A common iteration scheme would solve this fixed source problem via the process
\begin{equation}
 \psi_g^{k+1} = \mathcal{L}^{-1}_g  \mathcal{M}  \mathcal{S}_{gg}  \mathcal{D} \psi^{k}_g + \mathcal{L}^{-1}_g q_g \, ,
\end{equation}
where the iteration index $k$ is not to be confused with a time index.
Such iteration is called \emph{Richardson iteration} or more frequently 
in the context of neutron transport \emph{source iteration}.

Hence, transport solvers typically have the action of $\mathcal{L}^{-1}$
available.  To construct the action $y \gets \mathcal{L} x$, we can solve
the linear system
\begin{equation}
 \mathcal{L}^{-1} y = x \, 
\end{equation}
via Krylov methods that require only the action $\mathcal{L}^{-1}$. 
Alternatively, the individual $\mathcal{L}_g$ for each group could
be explicitly constructed, though of course such construction 
would potentially be repeated at every time step due to any material
changes. 

It is not clear whether explicitly constructing $\mathcal{A}$ or
constructing its action using an explicit or approximate action of
$\mathcal{L}_g$ is most economical.  As with most such questions,
the answer is likely highly dependent on the problem being solved and
the machine on which the problem is solved.  Worth noting is that
either method that does not employ the action of $\mathcal{L}^{-1}$ can
be considered a ``sweepless'' algorithm; such an algorithm is
the focus of the work of Davidson and Larsen \cite{davidson2009std},
though they discuss it in the context of a particular spatial 
discretization and solver.


\begin{exercises}

  %----------------------------------------------------------------------------%
  \item \textbf{Time-Dependent Diffusion}. 
    Derive the time-dependent, multigroup diffusion equations including 
    delayed neutron precursors.
  
\end{exercises}