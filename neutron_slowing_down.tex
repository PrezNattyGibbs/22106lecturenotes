\chapter{Neutron Slowing Down}
\label{lec:neutron_slowing_down}

In the last lecture, we reviewed neutron scattering kinematics in order 
to define the energy transfer probability $P(E\to E')$ and the 
energy-dependence of differential scattering 
cross sections. In this 
lecture, we'll use these cross sections to determine the energy dependence 
of the neutron flux as neutrons ``slow down'' by consecutive collisions.
We will limit ourselves to homogeneous media (for now) and use 
both a direct numerical approach and simpler, classical schemes.


\section*{Slowing Down in a Homogenous System}

Suppose that the medium of interest is infinite in spatial extent, i.e.,
it has no boundaries, and it is homogeneous in makeup.  Then, all 
spatial dependence vanishes, leading to
\begin{equation}
       \Sigma_t(E) \psi (E, \bm{\hat{\Omega}})
            = Q(E, \bm{\hat{\Omega}}) \, ,
\end{equation}
and integration over angle yields
\begin{equation}
    \begin{split}
      & \Sigma_t(E)  \phi ( E) \\
        &= \int_{4\pi} d\Omega \int_{4\pi} d\Omega' \int^{\infty}_{0} \Sigma_s(E',\bm{\hat{\Omega}}' \to  E,\bm{\hat{\Omega}}) \psi(E', \bm{\hat{\Omega}'}) dE' \\
        &+ \frac{\chi(E)}{k} \int^{\infty}_{0} \nu(E')\Sigma_f(E') \phi(E') dE' \\
        &+ S_{\text{ext}}( E) \, .
    \end{split}
    \label{eq:infmedintegrated}
\end{equation}
    
If scattering is assumed to be isotropic in the LAB system, then 
$\Sigma_s(E',\bm{\hat{\Omega}}' \to  E,\bm{\hat{\Omega}}) = \Sigma_s(E'\to E)/4\pi$ and
\begin{equation}
\begin{split}
      \Sigma_t(E)  \phi (E) &= \int^{\infty}_{0} \Sigma_s( E' \to  E) \phi(E') dE' \\
        &+ \frac{\chi(E)}{k} \int^{\infty}_{0} \nu(E')\Sigma_f(E') \phi(E') dE' \\
        &+ S_{\text{ext}}(E) \, ,
\end{split}
\label{eq:infmedspectrum}
\end{equation}
which is the infinite medium or spectrum equation.  Here, $\phi (E)$ is 
referred to as the \emph{spectrum}.
    
Between about 1 eV and 100 keV (the \emph{resonance region}, which 
depends on nuclide),  fission 
and inelastic scattering can be neglected, and the dominant 
interaction is elastic scattering.  Hence, the spectrum
equation (without an external source) simplifies to the 
the \emph{slowing-down equation}
\begin{equation}
     \Sigma_t (E) \phi(E) = 
       \sum^{N_{\text{nucl}}}_{k} \frac{1}{1-\alpha_k} 
          \int^{E/\alpha_k}_E N_k \sigma_{s,k} (E') \phi(E') \frac{dE'}{E'} \, ,
\label{eq:sde_general}
\end{equation}
where $\Sigma_t(E)$ is the total macroscopic cross section, $N_k$ is 
the number density of nuclide $k$,  $\sigma_{s,k}(E)$ is the microscopic
elastic scattering cross section of nuclide $k$, 
$\alpha_k = (A-1)^2/(A+1)^2$ for a nuclide $k$ of mass $A$ (in amu), 
and $N_{\text{nucl}}$ is the number of nuclides.
    
For the particular case of just one, purely-scattering nuclide, 
\EQ{eq:sde_general} simplifies to
\begin{equation}
     \Sigma_s (E) \phi(E) = 
       \frac{1}{1-\alpha} 
          \int^{E/\alpha}_E \Sigma_{s} (E') \phi(E') \frac{dE'}{E'} \, ,
    \label{eq:sde_one_nuclide}
\end{equation}
and, by substitution, one can verify the corresponding solution
is
\begin{equation}
     \phi(E) = \frac{C}{E\Sigma_s (E)} \, ,
\end{equation}
where $C$ is a constant defined by normalization.  Because 
$\Sigma_s(E)$ is usually almost constant, the 
\emph{slowing down spectrum} typically goes as $1/E$.
    
\section*{Direct Solution of the SDE}

Let $E\to E_i$, where $\Delta E$ is small enough 
that $\Sigma(E)$ is well-approximated by interpolation
between $E_i$ and $E_{i+1}$.  Then
\begin{equation}
     \Sigma_t (E_i) \phi(E_i) = 
       \sum_{j=i}^{N_E} \Sigma_s (E_j \to E_i) \phi(E_j) \Delta E_j
         + S(E_i) \, ,
    \label{eq:sde_discretized}
\end{equation}   
where an external source $S(E)$ has been included for generality,
$N_E$ is the number of discrete energy points used, and 
$\Delta E_j = E_{j+1} - E_{j}$.

For ``continuous'' treatment, the matrix implied by 
$\Sigma_s (E_j \to E_i)$ is far too large.  Instead,
we can evaluate its values on-the-fly by defining
\begin{equation}
      \Sigma_{s,j\to i} \approx 
        \left\{
           \begin{array}{l l}
               \frac{\Sigma_s(E_j)}{(1-\alpha)E_j} & \quad \alpha E_j \leq E_i \leq E_j \\
               0                                   & \quad \text{otherwise} \, .
            \end{array} 
        \right.
\label{eq:scattermatrixappx}
\end{equation}
Remember, this approximation is valid only for \emph{isotropic} 
(i.e., \emph{s-wave}) scattering in the COM system.  Typically, this is the 
case for all but the highest energies.
    
Substitution of \EQ{eq:scattermatrixappx} into \EQ{eq:sde_discretized}
leads to
\begin{equation}
     \Sigma_t (E_i) \phi(E_i) = 
       \sum_{j=i}^{N_E} \sum^{N_{\text{nucl}}}_{k=1} \frac{\sigma_s (E_j) \phi(E_j) \Delta E_j }{(1-\alpha_k)E_j} 
         + S(E_i) \, .
\label{eq:sdediscretized}
\end{equation}
Note that the sum over energy includes \emph{all} energies, but, in practice, the sum for 
some nuclides would be limited to a smaller number of energies for all 
but hydrogen.
        
This equation uses a one-sided \emph{Riemann sum} for the integral.  One
could also use a \emph{midpoint} or \emph{trapezoid} rule.  Knott and 
Yamamoto use the same form as Eq.~\ref{eq:sdediscretized} but substitute an average 
value $\bar{E}_j = \sqrt{E_j E_{j+1}}$ for $E_j$ in the 
denominator.
    
To solve Eq.~\ref{eq:sdediscretized}, rearrange to get
\begin{equation}
      \phi_i =
      \frac{ 
         \sum\limits_{j=i+1}^{N_E} 
         \sum\limits^{N_{\text{nucl}}}_{k=1} \frac{\sigma_s (E_j) \phi(E_j) \Delta E_j }{(1-\alpha_k)E_j} + S_j}
          {\Sigma_{t, i} - \Sigma_{s, i\to i}} \, ,
\end{equation}
where 
\begin{equation}
     \phi_N = \frac{S_N}{\Sigma_{t,N}-\Sigma_{s,N\to N}} \, .
\end{equation}
Alternatively, neglect any external source and set $\phi_N$ directly.
Note that the indexing used here is reversed from the standard 
use of $E_1$ indicating the largest energy.  Rather, $E_N$ is
the largest energy.
As basic implementation of the method described is provided in 
Listing~\ref{list:sde}.  The implementation is limited to 
a single species with constant cross sections.   
\lstset{language=Python,
        caption=Direct Solution of the SDE, 
        label=list:sde,
        morecomment=[l]{\%}}
\lstinputlisting{code/sde.py}
Although the approach outlined (and implemented) is technically sound, it is not 
very efficient.  The reader should consult Knott and Yamamoto (page 1031) 
for implementation ideas, especially for simplifying the construction 
of the in-scatter source term.   
    
    
\section*{Classical Resonance Approximations}

Although the numerical solution of the the slowing down equation is
straightforward, in principle, it remains relatively expensive 
to solve for homogeneous media and, when applied to heterogeneous systems,
analyses have historically been limited to small problems.  As an
alternative, several approximations can be made that lead to a direct,
analytical solution.

Consider again the slowing-down equation, 
\begin{equation}
\begin{split}
     \Bigg ( N_r \sigma_{t,r}(E) &+ 
             \sum_{k\neq r} N_k \sigma_{s,k} 
     \Bigg ) \phi(E) = \\
       & \frac{1}{1-\alpha_r} 
          \int^{E/\alpha_r}_E N_k \sigma_{s,r} (E') \phi(E') \frac{dE'}{E'} \\
       &+ \sum_{k\neq r} \frac{1}{1-\alpha_k} 
          \int^{E/\alpha_k}_E N_k \sigma_{s,k} \phi(E') \frac{dE'}{E'} \, ,
\end{split}          
\label{eq:sde_general_modified}
\end{equation}
modified such that terms related to a single resonator (identified by 
the $r$ subscript) are separate from all other nuclides $k$, and 
where the non-resonator cross sections are assumed to be independent
of energy and limited to elastic scattering, 
the first major approximations we shall make on our 
way to the narrow resonance (NR) and wide resonance (WR) approximations.

Next, we shall assume the ``practical'' width of a resonance is always
much smaller than the energy lost by a neutron scattering with all 
non-resonant nuclides.  Thus, for an integral of the form
\begin{equation}
 \frac{1}{1-\alpha_k} \int^{E/\alpha_k}_E N_k \sigma_{s,k} \phi(E') \frac{dE'}{E'} \, ,
\end{equation}
\emph{most} of the integration domain is far away from the resonance 
and, hence, we assume that the flux \emph{within the integral} 
takes its asymptotic form, i.e.,
the form found for a pure scatterer (with a constant 
cross section), $\phi(E)\propto 1/E$.  Therefore, 
the non-resonant scattering integral simplifies to
\begin{equation}
\begin{split}
  \frac{N_k \sigma_{s,k}}{1-\alpha_k}  
       \int^{E/\alpha_k}_E  \frac{1}{E'} \frac{dE'}{E'}
  &= \frac{N_k \sigma_{s,k}}{E} \, .
\end{split}          
\label{eq:simple_scattering_integral}
\end{equation}

\subsubsection*{Narrow Resonance Approximation}

The NR approximation further assumes that energy loss of a neutron 
scattering with the resonator is also much smaller than the 
resonance width, which means the same approximation can be made 
for the resonance scattering integral as was made for the 
non-resonant scattering angle.  With the additional assumption 
that $\sigma_{s, r}(E) \approx \sigma_{s, r}$, the slowing 
down equation simplies to
\begin{equation}
     \Bigg ( N_r \sigma_{t,r}(E) + 
             \sum_{k\neq r} N_k \sigma_{s,k} 
     \Bigg ) \phi(E) = \frac{\Sigma_{s,r}}{E} + \sum_k \frac{\Sigma_{s,k}}{E} \, ,
\end{equation}
and, hence, the spectrum in the narrow resonance approximation is
\begin{equation}
 \phi_{\text{NR}}(E) = 
   \frac{ \Sigma_{s,r} + \sum\limits_{k\neq r} \Sigma_{s,k} }{ \Sigma_{t}(E) E } \, ,
\end{equation}
with arbitrary normalization.

\subsubsection*{Wide Resonance Approximation}

Contrarily, the WR approximation assumes that the energy lost by a
neutron scattering off the resonant nuclide is much \emph{smaller} than 
the resonance width.  The smallest such energy loss 
occurs in the limit $\alpha_r \to 1$\footnote{i.e., when the 
resonator has an infinite mass, which explains why WR sometimes 
is called the narrow resonance, infinite mass (NRIM) approximation)} 
for which case the corresponding 
scattering integral simplifies to 
\begin{equation}
\begin{split}
 \lim_{\alpha_r \to 1} & \left [ \frac{1}{1-\alpha_r} 
          \int^{E/\alpha_r}_E N_k \sigma_{s,r} (E') \phi(E') \frac{dE'}{E'} \right ] \\ 
  &\approx N_k \sigma_{s,r} (E) \phi(E) \lim_{\alpha_r \to 1}  \int^{E/\alpha_r}_E  \frac{dE'}{(1-\alpha_r)E'} \\
  &=  N_k \sigma_{s,r} (E) \phi(E) \, .
\end{split}
\end{equation}
It follows that 
\begin{equation}
 \phi_{\text{WR}} = \frac{\sum\limits_{k\neq r} \Sigma_{s,k} }{[\Sigma_t(E) - \Sigma_{s,r}(E)]E } \, .
\end{equation}

\section*{Further Reading}

To be continued.


%------------------------------------------------------------------------------%
\begin{exercises}

  %----------------------------------------------------------------------------%
  \item \textbf{Slowing Down in Purely Scattering Media}. 
    Adapt the slowing-down code to treat a single medium with arbitrary
    mass number $A$.
    Using a point source at $E=10~\kilo\electronvolt$, compute the spectrum and plot 
    for $1~\electronvolt \leq E \leq 10~\kilo\electronvolt$.
    For $A=238$, zoom in and plot the rather strange (``Placzek'') 
    oscillations occuring 
    near the source energy.  Can you explain this behavior?
    
  %----------------------------------------------------------------------------%
  \item \textbf{Slowing Down in Arbitary Mixture}. 
    \label{prob:sdmix}
    Adapt the slowing-down code to treat an arbitary homogeneous
    mixture that uses real cross-section data from ENDF. Then
    \begin{enumerate}[(a)]
     \item Find total and elastic scattering cross-section data for 
           H-1 and U-238 (at room temperature) 
           and develop a way to define that data on 
           the same energy grid (sometimes called a ``unionized'' energy 
           grid).  
     \item Use the slowing-down code to determine the spectrum for 
           a 1000-to-1 mixture of H-to-U over 
           the range $E\in[1,100~\electronvolt]$ with a point source 
           at the upper limit.  Use at least 2000 energy points evenly
           spaced on a log scale.
     \item Plot the spectrum.  Repeat for 100-to-1 and 10-to-1 ratios 
           of H to U.
    \end{enumerate}

  %----------------------------------------------------------------------------%
  \item \textbf{The Narrow and Wide Resonance Approximations}. 
    For the same three cases studied in Problem~\ref{prob:sdmix}, determine 
    the spectrum for both the narrow and wide resonance approximations.
    For the 10-to-1 case, plot the numerical, NR, and WR spectra normalized
    so that $\phi(1~\electronvolt)$ is 1.

    
\end{exercises}