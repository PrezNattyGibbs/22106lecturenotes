\chapter{Neutron Scattering}

In the last lecture, we introduced the transport equation and 
its specialization to neutrons.  Over the course of the next several
lectures, we'll study ways to treat each of the phase-space 
variables, the first of which will be the energy variable.  However,
before we dive into an explicit treatment 
of the energy variable, we need to review how neutrons
change energy, i.e., how they slow down (or speed up).  Chiefly, neutrons 
slow down by elastic scattering, and this 
lecture provides a brief overview
of neutron scattering kinematics.

\section*{Scattering in the LAB Frame}

Suppose an incoming neutron of energy $E$ strikes a target of mass
$M$ at rest, as depicted in \FIGURE{fig:neutron_scatter}, where 
the outgoing neutron and target have energies $E'$ and $E_A$, 
respectively.  Our goal is to define a relationship between 
$E$ and $E'$ given the scattering angle $\theta$.

To relate $E$ to $E'$, we apply the laws of conservation of energy,
\begin{equation}
  E = E' + E_A \, ,
\label{eq:conservation_energy}
\end{equation}
and (linear) momentum,
\begin{equation}
\begin{split}
 p = p' \cos \theta  + p_A \cos \phi \\
 0 = p' \sin \theta  - p_A \sin \phi \, .
\end{split}
\end{equation}
This is a set of three equations in four unknowns ($E'$, $E_A$, 
$\theta$, and $\phi$).  To simplify, recall the law of 
cosines, i.e.,
\begin{equation}
  A^2 = B^2 + C^2 - 2BC\cos \alpha \, .
\end{equation}
Here, the momentum equations lead to a similar triangle, and, hence,
we have
\begin{equation}
  p_A^2 = (p')^2 + p^2 - 2p'p \cos \theta \, .
\end{equation}

Classically, the momentum and energy of a particle of 
mass $m$ are related by
\begin{equation}
  p^2 /2m = E \, .
\end{equation}
Hence,
\begin{equation}
 2 m_A E_A = 2 m E' + 2 m E - 4m \sqrt{E E'} \cos \theta \, ,
\end{equation}
or, by using \EQ{eq:conservation_energy}, we have 
\begin{equation}
 m_A (E-E') = m (E' + E - 2 \sqrt{EE'}\cos \theta) \, ,
\end{equation}
and rearranging the result gives
\begin{equation}
  0 = E'(A+1) - \sqrt{E'} (2 \sqrt{E} \cos \theta ) - E (A-1) \, ,
\end{equation}
which is quadratic in $\sqrt{E'}$, and where $A=m_A/m$.  
Solving for $E'$ and 
simplifying leads to
\begin{equation}
 E' = \frac{E}{(A+1)^2} (\cos \theta + \sqrt{A^2 - \sin^2 \theta})^2 \, .
\label{eq:scattered_neutron_energy_lab}
\end{equation}

For a ``grazing'' collision, $\theta \approx 0$, 
and \EQ{eq:scattered_neutron_energy_lab} gives
\begin{equation}
  E' \approx \frac{E}{(A+1)^2} (1 + \sqrt{A^2})^2 \approx E \, ,
\end{equation}
as expected.  

The minimum outgoing energy (for $A > 1$) corresponds to backward scattering,
i.e., $\theta = \pi$, so that
\begin{equation}
  E'_{\text{min}} = E \left ( \frac{A-1}{A+1} \right )^2 = \alpha E \, .
\end{equation}
Although this expression is valid for $A = 1$, i.e., when hydrogen is 
the target, the minimum occurs for $\theta = \pi/2$.

\section*{Scattering in the COM Frame}

So we have $E'$ in a one-to-one correspondence with $\theta$, but these 
values are in the {\emph laboratory} frame of reference.  Some 
analysis is a bit simpler in the center-of-mass frame, so we really
want to relate $\theta$ to $\theta_c$ and $E'$ to $\theta_c$.

In \FIGURE{fig:frames_of_reference}, $v_c$ and $V_c$ are the 
velocities of the neutron and target in the center-of-mass system.

The total system in each coordinate system must exhibit conserved 
momentum, i.e.,
\begin{equation}
   (m + m_A) V_cm = mv + m_A V_a \, .
\end{equation}
In the LAB frame, the target is stationary so that
\begin{equation}
  V_cm = \frac{mv}{m + m_A} = \frac{v}{1+A} \, .
\end{equation}
To translate particle velocities from LAB to COM values
requires we subtract $V_{cm}$ from the LAB values, i.e.,
\begin{equation}
 v_c = v - V_{cm} = \frac{A}{1+A}v
\end{equation}
and
\begin{equation}
 V_c = -V_{cm} = -\frac{1}{1+A}v
\end{equation}

We now use these values in conservation equations to find 
the neutron energy before and after the collision, i.e.,
\begin{equation}
 E_c = \frac{1}{2} m v_c^2 \frac{1+A}{A} \, 
\end{equation}
and
\begin{equation}
 E'_c = \frac{1}{2} m (v'_c)^2 \frac{1+A}{A} \, .
\end{equation}
It follows that
\begin{equation}
 v_c = v_c' = \frac{A}{1+A} v
\end{equation}
and
\begin{equation}
 V_c = \frac{1}{1+A}v \, .
\end{equation}

Now, we look to develop relationships between (1) $\theta_c$ 
and $\theta$ and (2) $E'$ and $\theta_c$.
Note that
\begin{equation}
 v'_c \sin \theta_c = v' \sin \theta
\label{eq:lab_com_1}
\end{equation}
and 
\begin{equation}
 v'_c \cos \theta_c + v_{cm} = v' \cos \theta
\label{eq:lab_com_2}
\end{equation}
from which it follows that
\begin{equation}
  \tan{\theta} = \frac{\sin \theta_c}{\cos \theta_c + \gamma} \, ,
\end{equation}
where 
\begin{equation}
 \gamma = 1/A \, .
\end{equation}
Then, one can show that 
\begin{equation}
 \cos \theta = \frac{  A \cos \theta_c + 1 } {\sqrt{ A^2 + 2 A \cos \theta_c +1 }} \, ,
\label{eq:theta_thetac}
\end{equation}
proof of which is left as an exercise to the student.


Then, by squaring and adding \EQS{eq:lab_com_1}~and~\ref{eq:lab_com_2}, 
one can show that 
\begin{equation}
 \left( \frac{v'}{v}\right )^2 = \frac{E'}{E} = \frac{A^2 + 2A \cos \theta_c +1 }{(1+A)^2}
\end{equation}
or 
\begin{equation}
 \frac{E'}{E}  = \frac{1}{2} [1+\alpha + (1-\alpha)\cos \theta_c] \, .
\label{eq:scattered_neutron_energy_com}
\end{equation}

\section*{Scattering Cross Sections}

In the transport equation for neutrons, we encountered the 
macroscopic \emph{double-differential scattering cross 
section} $\Sigma_s(E\to E', \Omega\to \Omega')$, where 
the spatial variable is suppressed.   Let's consider 
the microscopic, double-differential scattering cross section
\begin{equation}
 \sigma_s(E\to E', \Omega\to \Omega') \, .
\end{equation}
By integrating over either $\Omega'$ or $E'$, we have 
the (single) differential scattering cross sections
\begin{equation}
 \sigma_s(E\to E') = \int_{4\pi} \sigma_s(E\to E', \Omega\to \Omega') d\Omega'
\end{equation}
and 
\begin{equation}
  \sigma_s(E, \Omega\to\Omega') = \int^{\infty}_0 \sigma_s(E\to E', \Omega\to \Omega') dE' \, ,
\end{equation}
or the (total) scattering cross section
\begin{equation}
 \sigma_s(E) = \int^{\infty}_0 \int_{4\pi} \sigma_s(E\to E', \Omega\to \Omega') d\Omega' dE' \, .
\end{equation}

We can relate the differential scattering cross section
$\sigma_s(E\to E')$ to the total scattering cross section
$\sigma_s(E)$ with the form 
\begin{equation}
  \sigma_s(E\to E') = \sigma_s(E)P(E\to E') \, ,
\end{equation}
where $P(E\to E')$ is, loosely, the probability 
that a neutron of energy $E$ scatters to energy $E'$.
For elastic scattering, we can define $P(E\to E')$ explicitly,
but for inelastic scattering, $P(E\to E')$ must be defined through 
experimental measurements.

Consider again the differential scattering cross section
$\sigma_s(\Omega \to \Omega')$.
If we assume the material in question is isotropic, i.e., that the 
scattering of neutrons does not depend on the orientation of 
the neutron's direction of travel with respect to the material,
then the scattering cross section depends only on the scattering 
angle, or 
\begin{equation}
\begin{split}
 \sigma_s(\Omega \to \Omega') &= \sigma_s(E\to E', \Omega\cdot \Omega') \\
                              &= \sigma_s(E\to E', \cos\theta)/2\pi \\
                              &= \sigma_s(E\to E', \mu)/2\pi \, ,
\end{split}
\end{equation}
where
\begin{equation}
 \mu = \cos\theta \, ,
\end{equation}
and, hence,
\begin{equation}
 d\mu = \sin\theta d\theta \, .
\end{equation}
The factor of $2\pi$ comes from the fact that we must have
\begin{equation}
 \int_{4\pi} \sigma_s(\Omega \to \Omega')  d\Omega'
   = \int^{2\pi}_0 \int^{1}_{-1}\frac{\sigma_s(E\to E', \mu)}{2\pi}
     d\mu d\phi \, .
\end{equation}
We can use these expressions to define the probability 
$P(\Omega\cdot \Omega')$, i.e., the probability that 
a neutron scatters through the angle $\theta$, or
\begin{equation}
 P(\theta) 2\pi \sin \theta d\theta = \frac{\sigma_s(\theta)}{\sigma_s}  2\pi \sin \theta d\theta \, .
\end{equation}
Because there is a one-to-one relationship between the outgoing 
energy $E'$ and the scattering angle $\theta$ (either in the LAB or 
in the COM frames), we can relate
the transfer probability $P(E\to E')$ to one for the scattering angle, 
i.e.,
\begin{equation}
 P(E\to E') dE' = - \frac{\sigma_s(\theta)}{\sigma_s} 2\pi \sin \theta d\theta \, .
\end{equation}

In the COM system, elastic scattering is isotropic for all
but the highest energies, i.e.,
\begin{equation}
  \sigma(\theta_c) \approx \text{constant} \equiv \frac{\sigma_{s}}{4\pi} \, .
\end{equation}
Then, from \EQ{eq:scattered_neutron_energy_com}, we have
\begin{equation}
 \left | \frac{dE'}{d\theta_c} \right | = \frac{ E (1-\alpha) \sin \theta_c}{2} \, ,
\end{equation}
and, hence, 
\begin{equation}
      P(E\to E') =
        \left\{
           \begin{array}{l l}
               \frac{1}{E(1-\alpha)} & \quad \alpha E \leq E' \leq E \\
               0                     & \quad \text{otherwise} \, .
            \end{array} 
        \right.
    \label{eq:scattermatrixappx}
\end{equation}

\section*{Further Reading}

Most of this development can be found in similar form in any good 
reactor physics text, e.g., Duderstadt and Hamilton\cite{duderstadt1976nra}.


%------------------------------------------------------------------------------%
\begin{exercises}

  %----------------------------------------------------------------------------%
  \item \textbf{Outgoing Energy in the LAB}. 
    Prove \EQ{eq:scattered_neutron_energy_lab}

  %----------------------------------------------------------------------------%
  \item \textbf{Relating LAB and COM}. 
    Prove \EQ{eq:theta_thetac}
 
  %----------------------------------------------------------------------------%
  \item \textbf{Outgoing Energy in the COM}. 
    Prove \EQ{eq:scattered_neutron_energy_com}
    
  %----------------------------------------------------------------------------%
  \item \textbf{The Average Outgoing Energy}. 
    Prove that the average outgoing energy (in LAB) 
    after a collision from $E\to E'$
    is 
    \begin{equation*}
      \bar{E}' = \frac{1+\alpha}{2} E \, .
    \end{equation*}

 %----------------------------------------------------------------------------%
  \item \textbf{Average Cosine of the Scattering Angle}. 
    Prove that, for the case of isotropic scattering in the COM frame, 
    the average cosine of the LAB scattering angle is given by
    \begin{equation*}
      \bar{\mu} = \frac{2}{3A} \, .
    \end{equation*}  
    Hence, in the LAB frame, neutron scattering tends to be anisotropic 
    with a forward bias.

 %----------------------------------------------------------------------------%
  \item \textbf{Average Logarithmic Energy Loss}. 
    A useful quantity is the so-called ``lethargy,'' defined as
    \begin{equation}
     u = \ln E_0 / E \, 
    \end{equation}
    for some reference energy $E_0$ (usually a large value like 10~MeV).
    Show that the average logarithmic energy loss is given by
    \begin{equation}
     \xi = \braket{\ln(E/E')} = 1 + \frac{\alpha}{1-\alpha}\ln{\alpha} \, .
    \end{equation}
    

 %----------------------------------------------------------------------------%
  \item \textbf{Average Logarithmic Energy Loss for Large $A$}. 
    Show that $\xi \approx 2/(A+2/3)$ for large $A$.
   
  %----------------------------------------------------------------------------%
  \item \textbf{Anisotropic Scattering}. 
    Suppose that 
    \begin{equation}
       \sigma(\theta_c) = \frac{\sigma_s}{4\pi}(1 + a \cos \theta_c) \, ,
    \end{equation}
    where $\theta_c$ is the scattering angle in the COM frame.  
    \begin{enumerate}[(a)]
     \item Determine $P(E\to E')$ and sketch it for $\alpha E \leq E' \leq E$ for 
           the case of $a > 0$, $a = 0$, and $a < 0$.  
     \item Determine $P(\theta)$ (i.e., for the LAB), and plot over 
           $0 \leq \pi$.
    \end{enumerate}

    
\end{exercises}
