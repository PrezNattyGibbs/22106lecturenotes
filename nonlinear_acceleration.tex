\chapter{Nonlinear Acceleration}
\label{lec:nonlinear_acceleration}

The discrete ordinates method is in theory a very fast, memory-efficient 
technique.  Unless the angular flux is needed explicitly, only one set of edge 
angular fluxes is needed at a time (e.g. the left edge flux for computing the 
right edge flux, which then can overwrite the left edge flux).  

Despite the algorithmic simplicity of the method, the underlying source 
iteration scheme can be obnoxiously slow for problems with high scattering 
ratios.  Several methods have been introduced over the years to accelerate the 
source convergence.  Of these, diffusion synthetic acceleration (DSA) is 
probably the most powerful and widespread.  The method essentially involves 
performing a diffusion solve to update the scalar flux, and has shown 
spectacular success for a wide variety of problems.  DSA and related 
methods will be cast as ``preconditioners'' for linear solvers
in the next lecture.

Here, we look at two  relatively simple 
methods that have been quite successful: the \textit{coarse mesh rebalance} 
(CMR) and \textit{Nonlinear Diffusion Acceleration} (NDA) methods.  
Although the presentation is restricted to the one-group case, both 
methods apply readily to multigroup fixed-source and eigenvalue problems.
In addition, both methods can be applied successfully to other 
transport discretization, including the method of characteristics.

\section*{Coarse Mesh Rebalance}

CMR was probably the first wide-spread method for accelerating discrete 
ordinates calculations.  Its foundation rests in the \textit{neutron balance 
equation}, which we obtain in one dimension by integrating
\EQ{eq:slabtransportequation} over angle, leading to
\begin{equation}
\begin{split}
 \frac{\partial }{\partial x}J(x) + \Sigma_t(x) \phi(x) 
   = \Sigma_{s}(x)\phi(x) + S(x) \, ,
\end{split}
\label{eq:balance}
\end{equation}
where $J = \mathbf{J}\cdot \hat{i}$ is the net current in the $x$ direction. Now 
suppose we divide the domain into a number of coarse meshes, indexed by $j$, 
with cell edges at $x_{j-\frac{1}{2}}$ and $x_{j+\frac{1}{2}}$.  Subtracting the 
scattering source from both sides and integrating over the $j$th coarse mesh, we 
obtain
\begin{equation}
\begin{split}
 J_{j+\frac{1}{2}} - J_{j-\frac{1}{2}}  + \int_{x_i} \Sigma_r(x) \phi(x) dx = 
\int_{x_i} S(x) dx \, ,
\end{split}
\label{eq:coarsebalance}
\end{equation}
where $\Sigma_t - \Sigma_s$ is the \textit{removal} cross section.
Expressing \EQ{eq:net2partial} as $J = J^+ - J^-$, where $+$ indicates to 
the right and $-$ to the left, we have
\begin{equation}
\begin{split}
 J^+_{j+\frac{1}{2}}- J^-_{j+\frac{1}{2}} - J^+_{j-\frac{1}{2}} + 
J^-_{j-\frac{1}{2}} + R_j = S_j \, ,
\end{split}
\label{eq:cmrbalance}
\end{equation}
where the partial currents are defined
\begin{equation}
 J^{\pm}_{j+\frac{1}{2}} = \sum_{\mu_n \gtrless 0} \mu_n \psi_{j+\frac{1}{2},n} 
\, ,
\end{equation}
$R_j$ is the coarse mesh-integrated removal rate, and $S_j$ is the coarse 
mesh-integrated source.  

\EQUATION{eq:cmrbalance} must be satisfied by a fully converged fine mesh 
solution.  For an unconverged iterate, which we denote $\psi^{m+\frac{1}{2}}$, 
\EQ{eq:cmrbalance} is generally not satisfied.  To force the flux to 
satisfy neutron balance, we introduce multiplicative rebalance factors $f_j$ 
such that
\begin{equation}
 \psi^{m+1}_{i,n} =
 \begin{cases} f_j \psi^{m+\frac{1}{2}}_{i,n}     &   x_{j-\frac{1}{2}} < x_{i} 
< x_{j+\frac{1}{2}} \\
               f_{j-1} \psi^{m+\frac{1}{2}}_{i,n} &   x_{i} = x_{j+\frac{1}{2}} 
\text{ and } \mu_n > 0 \\
               f_{j+1} \psi^{m+\frac{1}{2}}_{i,n} &   x_{i} = x_{j+\frac{1}{2}} 
\text{ and } \mu_n < 0 \\
 \end{cases} \, .
 \label{eq:alpha}
\end{equation}
Note the appropriate factor's index denotes from which coarse mesh the neutrons 
originate. For the case of vacuum boundaries, the corresponding incident partial 
current vanishes.  For a reflective boundary at the left, $f_0 = f_1$, and 
similarly for the right boundary.

To illustrate the method, consider a slab with vacuum boundaries divided into 
three coarse meshes.  Substituting the modified fluxes into the balance equation 
yields a set of three equations:
\begin{equation}
\begin{split}
  f_1 J^+_{3/2}  - f_2 J^{-}_{3/2}                   + f_1 J^{-}_{1/2} + f_1 R_1 
&= f_1 S_1 \\
  f_2 J^+_{5/2}  - f_3 J^{-}_{5/2} - f_1 J^{+}_{3/2} + f_2 J^{-}_{3/2} + f_2 R_2 
&= f_2 S_2 \\
  f_3 J^+_{7/2}                    - f_2 J^{+}_{5/2} + f_3 J^{-}_{5/2} + f_3 R_3 
&= f_3 S_3 \, ,
\end{split}
\end{equation}
which yields a tridiagonal system written in condensed form as
\begin{equation}
 \mathbf{A} \mathbf{f} = \mathbf{S} \, .
\end{equation}
Solving this equation for $\mathbf{f}$ and updating to $\psi^{m+1}$ allows us to 
recompute the scattering source, which---hopefully---converges faster than 
without the rebalancing scheme.  Look to the Further Reading section for 
references that investigate stability of coarse mesh rebalance.  
Algorithm~\ref{alg:accel} shows how CMR (or any low order acceleration scheme) 
can be used within source iteration.

\begin{algorithm}
 \label{alg:accel}
 \caption{Accelerated Source Iteration}
  initialization\;
  \While{$\psi^m$ or $\phi^m$ not converged}{
    compute scattering source \;
    $\psi^{m+\frac{1}{2}} \leftarrow \text{sweep}(\psi^{m})$ \;
    \eIf{using CMR}{
      $\psi^{m+1} \leftarrow \text{cmr}(\psi^{m+\frac{1}{2}})$ \;
    }{
      $\psi^{m+1} \leftarrow \psi^{m+\frac{1}{2}}$ \;
    }
  }
\end{algorithm}



\section*{Nonlinear Diffusion Acceleration}

Nonlinear Diffusion (or Coarse-Mesh, Finite-Difference) Acceleration (NDA
or CMFD) was originally developed by Kord Smith as a way to reduce the 
memory requirements of advanced, nodal-diffusion solvers \cite{smith1983nms}.  
The method also
led to signficant speed-ups and has since been applied routinely in 
for lattice-physics computations based on the method of characteristics
\cite{2002}.  

We begin along the same lines as we did for CMR by integrating the 
transport equation over a coarse mesh indexed by $j$, which leads to 
\EQ{eq:coarsebalance}.  Now, define the flux-weighted cross section
\begin{equation}
 \Sigma_{t,j} = \frac{\int^{x_{j+1/2}}_{x_{j-1/2}} \Sigma_r(x) \phi^{n+1/2} dx}
                     {\int^{x_{j+1/2}}_{x_{j-1/2}}\phi^{n+1/2} dx} \, ,
\end{equation}
with similar expressions for $\Sigma_{r,j}$ and $\Sigma_{s,j}$, along 
with the coarse-mesh-averaged flux
\begin{equation}
 \phi_{j} = \frac{\int^{x_{j+1/2}}_{x_{j-1/2}} \phi(x) dx}
              {\int^{x_{j+1/2}}_{x_{j-1/2}} dx} \, 
\end{equation}
and source
\begin{equation}
 S_{j} = \frac{\int^{x_{j+1/2}}_{x_{j-1/2}} S(x) dx}
              {\int^{x_{j+1/2}}_{x_{j-1/2}} dx} \, .
\end{equation} 
With these definitions, \EQ{eq:coarsebalance} can be written as
\begin{equation}
 J^{n+1/2}_{j+1/2} - J^{n+1/2}_{j-1/2} 
   + \Delta_j \Sigma_{r,j} \phi_j = \Delta_j S_j
\end{equation}

Previously, we studied fine-mesh diffusion, which gave
\begin{equation}
\begin{split}
  J^{n+1/2}_{j+1/2} &= - \left ( \frac{ 2 D_j D_{j+1} }
                                     {D_j \Delta_{j+1} + D_{j+1} \Delta_j}
                        \right )
                        (\phi_{j+1} - \phi_j) \\\
                    &= \tilde{D}_{j+1/2} (\phi_{j+1} - \phi_j) \, .
\end{split}                    
\label{eq:diffusion_current}
\end{equation}
where $D_j$ is the diffusion coefficient for the $j$th coarse cell.
Although the choice of $D_j$ is not unique, a common approach is 
to set 
\begin{equation}
 D_j = 1/3\Sigma_{t, j}
\end{equation}
or 
\begin{equation}
 D_{j} = \frac{\int^{x_{j+1/2}}_{x_{j-1/2}} D(x) \phi(x) dx}
              {\int^{x_{j+1/2}}_{x_{j-1/2}} \phi(x) dx} \, .
\end{equation}
A better approach may be to weight $D(x)$ by $J(x)$ instead of 
$\phi(x)$, but in practice $J(x)$ is not usually computed.

Even with a flux- or current-weighted $D_j$, the current as 
defined by \EQ{eq:diffusion_current} is not generally 
equal to the current as computed directly from 
$\psi^{n+1/2}$, as in Eq. XXX, because the diffusion approximation is 
not guaranteed to yield the same solution as a transport 
approximation.  Hence, even if the flux were perfectly converged, 
the diffusion approximation, coarse mesh or otherwise, will not 
reproduce the solution and, hence, does not preserve balance over
the coarse cells.

We can enforce balance by letting
\begin{equation}
 J_{j+1/2} = -\tilde{D}_{j+1/2} (\phi_{j+1}-\phi_j) - \hat{D}_{j+1/2}(\phi_{j+1}+\phi_j) \, ,
\end{equation}
where the nonlinear coupling coefficient $\hat{D}_{j+1/2}$ is 
defined as
\begin{equation}
 \hat{D}_{j+1/2} = \frac{\tilde{D}_{j+1/2} (\phi_{j+1}-\phi_j) + J_{j+1/2}}{\phi_{j+1}+\phi_j)} \, .
\end{equation}
For an albedo condition at the left, we can set 
\begin{equation}
 J_{1/2} = (\tilde{D}_{1/2}+\hat{D}_{1/2})\phi_1 \, ,
\end{equation}
where 
\begin{equation}
 \hat{D}_{1/2} = \frac{J_{1/2}-\tilde{D}_{1/2} \phi_1}{\phi_1} \, .
\end{equation}
The resulting, one-dimensional diffusion equation is
\begin{equation}
\begin{split}
 -\tilde{D}_{j+1/2}(\phi_{j+1}-\phi_j) - \hat{D}_{j+1/2} & (\phi_{j+1}+\phi_j)  + \\
 \tilde{D}_{j-1/2}(\phi_{j}-\phi_{j-1})& + \hat{D}_{j-1/2} (\phi_{j}+\phi_{j-1}) + \\
  & \Sigma_{r,j} \phi_{j}^{n+1} \Delta_j = S_j \Delta_j \, .
\end{split}
\end{equation}



Once solved, the coarse-mesh flux $\phi^n$ can be used to 
update the fine-mesh flux via 
\begin{equation}
 \phi^{n+1}(x) = \frac{\phi^{n}_{j}}{\phi^{n-1/2}_j} \phi(x) \, \quad x \in [x_{j-1/2},x_{j+1/2}] \, .
\end{equation}
With $\phi^{n}(x)$ found, the scattering source can be updated,  another 
sweep can be performed to obtain $\psi^{n+3/2}$, and the
process can be repeated. 


Like CMR, if the NDA converges, it must converge to 
the right answer.  However, that convergence is not guaranteed \cite{}.
Sometimes, for a given $\tilde{D}$, the computed $\hat{D}$ leads to 
a diffusion matrix that is not diagonally dominant.  While this does 
not prevent solution of the equations, the updated, coarse-mesh flux 
may contain unphysical features that lead to instability.  
Specifically, if $\hat{D}_{j+1/2} > \tilde{D}_{j+1/2}$, setting
\begin{equation}
 \hat{D}_{j+1/2} = \tilde{D}_{j+1/2} = \frac{J_{j+1/2}}{2\phi_j} \, 
\end{equation}
ensures a diagonally-dominant matrix.

In addition, the use of $D_j$ directly may be replaced by 
Larsen's ``effective'' diffusion coefficient, which may be particularly
valuable for the step-characteristic scheme in use for both MOC and S$_N$
calculations.  Even though $D_j$ is arbitary, its proper selection can 
help convergence.


Finally, it has been found that instabilities can be reduced 
significantly by using under-relaxation of the 
nonlinear coupling coefficient \cite{}.  In other words, 
the coefficients are defined
\begin{equation}
 \hat{D}^{n+1}_{j+1/2} = \alpha \hat{D}^{n+1}_{j+1/2} + (1-\alpha)\hat{D}^{n}_{j+1/2}  \, ,
\end{equation}
where  $\alpha \in [0, 1)$ is the relaxation parameter.  Based on 
past experience, typical values for $\alpha$ range from 0.5 to 0.8, 
though the optimum value is highly problem dependent.





\section*{Further Reading}

A review paper by Adams and Larsen \cite{adams2002fim} provides a survey of the 
many acceleration techniques available for the discrete ordinates equations, and 
with its several hundred references, is the place to look for further 
information.  Lewis and Miller provides more information on coarse mesh 
rebalance, and Cefus and Larsen have assessed its stability \cite{cefus1990sac}. 
 Park and Cho \cite{park2004cma} have suggested angular-dependent rebalance 
schemes, and their work is a good place to start to find references to other CMR 
variations.  The discontinuity factors used in CMFD were first proposed by Smith 
in the context of homogenization \cite{smith1980shm}, and he later proposed 
their use for acceleration \cite{smith1983nms}. A recent paper by Zhong et al. 
\cite{zhong2008itl} provides a modern overview and advanced use of the approach 
along with a relatively detailed set of equations for implementation.  


%  For edge $j+\frac{1}{2}$ joining meshes $j$ and $j+1$, we compute from the high order solution the net current $J_{j+\frac{1}{2}}$ and enforce
% \begin{equation}
%  J_{j+\frac{1}{2}} = -\tilde{D}_{1+\frac{1}{2}} \Big ( \bar{\phi}_{j+1}-\bar{\phi}_{j} \Big ) -\hat{D}_{1+\frac{1}{2}} \Big ( \bar{\phi}_{j+1}-\bar{\phi}_{j} \Big ) \, ,
% \end{equation}
% where
% \begin{equation}
%  \D
% \end{equation}
