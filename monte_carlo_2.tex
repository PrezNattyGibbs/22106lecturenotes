\chapter{Random Numbers and Collision Physics}

In order to simulate particle transport stochastically, a way 
to sample the ``random'' nature of particles is needed.  In this 
lecture, the generation of pseudorandom number sequences is 
presented.  With these sequences, it is possible to sample 
from arbitrary probability densities.  
Finally, a basic particle transport algorithm is proposed.


% -----------------------------------------------------------------------------
\section*{Random Numbers}

The heading is a misnomer---\emph{numbers} are not random, but a 
\emph{sequence} of numbers can be random. Such sequences can be 
created using the natural variation of our 
surroundings, which could include the number of decays counted in 
a specific interval of time or the noise on an electrical 
system \footnote{In fact, random.org provides random sequences
based on atmospheric noise.}.  

For scientific simulations, truly random sequences are not always desirable, 
specifically  when such simulations are to be repeated for debugging purposes.
As an alternative to truly random sequences,  
\emph{pseudorandom} sequences are often used.  A pseudorandom 
sequence is a deterministic sequence, which means that given an 
initial \emph{seed}, every number in the sequence can be 
found directly (although not always efficiently).  Moreover,
pseudorandom sequences (at least good ones) exhibit statistical
properties similar to those expected of truly random sequences.



\emph{Linear Congruential Generators} (LCG's) are probably 
the most common way to generate psuedorandom 
sequences in practice and the only ones 
to be described here.  A LCG is defined by the sequence
\begin{equation}
  S_{i+1} = \Big ( S_{i} \cdot g+c \Big ) \: \textrm{mod} \: p
\label{eq:lcg}
\end{equation}
where $S_0$ is the  \emph{seed},
$g$ is the   \emph{generator} or   \emph{multiplier},
$c$ is the  \emph{adder} or \emph{increment},
and $p$ is the  \emph{modulus}.  

Use of a LCG is straightforward.  An initial, integer seed ($S_0$)
is selected, with which any subsequent value 
can be found by using \EQ{eq:lcg}  
However, the numbers from the psuedorandom sequence are most often used to 
produce a number $r$ drawn uniformly from the range 0 to 1.  Hence,
the number $S_k$ must be divided by the modulus to 
get $r_k = S_k / p$.


The utility of a LCG depends strongly on the parameter values 
selected.  A typical choice for the modulus is to 
set $p = 2^m$ for some integer $m$.  This choice can simplify 
evaluation of the modulo operator (on some systems) 
by discarding $m$ largest digits.  Usually,
$m$ selected as number of bits in the largest integer.
The \emph{period} of such LCG's is limited to $\frac{p}{4}$ or $2^{m-2}$,
where the period is the smallest integer $n$ such that $S_0 = S_n$.


\textbf{Example}: $p=2^5$, $g=21$, $c=1$, and $s_0 = 13$
\begin{equation*}
        s_1 = (13 \cdot 21 +1) \, \textrm{mod} \: 2^5 
            =  \underbrace{274}_{\cancel{1000}\textbf{10010}} \: \textrm{mod} \, 2^5 = \underbrace{18}_{\textbf{10010}} \, .
\end{equation*}


\emph{Choosing the Multiplier and Increment}
  \begin{itemize}
     \item Choose $g$ and $c$ to maximize the period
     \item Larger values of the $g$ help reduce correlation
     \item $c=0$ produces a maximum possible period of $2^{m-2}$
     \item $c=1$ produces a maximum possible period of $2^m$
  \end{itemize}
% c=0 is multiplicative, c=1 is mixed


\emph{Typical LCG values}
  \begin{center}
  \begin{tabular}{ | l | c | c | c | c | }
\hline
  Code & p & period & g & c \\
\hline
  MCNP5 (LANL) & $2^{63}$ & $2^{63}$ & variable & $1$ \\
  RACER (KAPL) & $2^{47}$ & $2^{45}$ & $84000335758957$ & $0$ \\
  VIM (ANL) & $2^{48}$ & $2^{46}$ & $5^{19}$ & $0$ \\
  MCNP4 (LANL) & $2^{48}$ & $2^{46}$ & $5^{19}$ & $0$ \\
\hline   
  \end{tabular}
  \end{center}


\emph{LCG testing}
  \begin{itemize}
    \item Spectral test (want to minimize the distance between hyper-planes)
    \item Diehard test suite (G. Marsaglia)  
  \end{itemize}


\section{Sampling}

\subsection{Direct Sampling}

\emph{Transformation of RV's}
 
Suppose $x$ is a RV with pdf $f(x)$ and a function $y(x)$
has a pdf $g(y)$.  Then
\begin{equation*}
  |g(y)dy| = |f(x)dx| \quad \text{or} \quad  g(y) = f(x) \left | \frac{dx}{dy} \right |
\end{equation*}

\vfill
\textcolor{mitred}{Special case: $y(x) = F(x)$}.  Then 
\begin{equation*}
 g(F) =  \left \{ 
         \begin{array}{l l}
           1      & 0 \leq F \leq 1 \\
           0      & \text{otherwise} \\
  \end{array} \right . 
\end{equation*}
\vfill
 $F(x)$ is a uniformly distributed RV


\emph{Direct Sampling}

  \begin{itemize}
     \item Direct inversion of the cdf
     \begin{itemize}
      \item Sample: $\xi \in (0, 1)$
      \item Set: $F(x)=\xi$
      \item Invert cdf: $x=F^{-1}(\xi)$
     \end{itemize}
%     \item Equivalent to Mathematical approach of the Monte Carlo integration
%Insert equation of x = F-1

     \item Advantages
        \begin{itemize}
           \item Often efficient (when $F^{-1}$ exists)
        \end{itemize}

     \item Disadvantages
        \begin{itemize}
           \item $F^{-1}$ may be expensive
           \item $F^{-1}$ may not exist (e.g. Klein-Nishina)
        \end{itemize}
  \end{itemize}


\emph{Direct Sampling Example}
  Consider the exponential distribution
  \begin{equation*}
     p(x)dx = \Sigma_{t} e^{-\Sigma_{t} x} \; \textrm{for $0 \le x \le \infty $}
  \end{equation*}
  with corresponding cdf
  \begin{equation*}
     F(x) = \int_{0}^{x} p(x')dx' = 1 - e^{-\Sigma_{t} x}
  \end{equation*}

 Sample via:
  \begin{equation*}
    \xi = 1 - e^{-\Sigma_{t} x}
  \end{equation*}
  \begin{equation*}
    e^{-\Sigma_{t} x} = 1 - \xi
  \end{equation*}
  \begin{equation*}
    x = -ln(1-\xi) / \Sigma_{t} \equiv -ln(\xi) / \Sigma_{t}
  \end{equation*}
%show path length sampling


\subsection{Rejection Sampling}
\emph{Rejection Sampling}
   \begin{itemize}
       \item Random sampling of $f(x)$ for which $F(x)$ is hard to invert
       \item Equivalent to the ``simulation'' approach of MC integration
   
       \item Advantages
          \begin{itemize}
              \item Very simple
          \end{itemize}
   
       \item Disadvantages
          \begin{itemize}
              \item Can become costly and/or confusing
          \end{itemize}         
   \end{itemize}


\emph[fragile]{Rejection Sampling Algorithm}

\begin{algorithm}[H]
    select $g(x)$ and $M$ such that $f(x) \leq Mg(x)$\;
    \While{$x$ is not found}{
      sample $x$ from $g(x)$ \;
      sample $\xi$ from $U(0, 1)$\;
      \uIf {$\xi <\frac{f(x)}{Mg(x)}$}{
        accept $x$\;
      }
      \Else {
        reject $x$\;
      }
    }
\end{algorithm}



\emph{}
\begin{figure}[htbp]
  \centering
%  \includegraphics[keepaspectratio, width = 3.5 in]{rejection_sampling}
\end{figure}


\emph{Discrete Sampling}
 \begin{itemize}
  \item Build cdf from discrete pdf
  \item Pick a random number $\xi$
  \item Perform table search
  \begin{itemize}
   \item Linear table search: go through each case one by one
   \begin{itemize}
    \item Worst case performance: $O(N)$
   \end{itemize}
   \item Binary search: Start in the middle of the number of bins and reduce search space in half each time
   \begin{itemize}
    \item Worst case performance: $O(log_{2}N)$
   \end{itemize}
  \end{itemize}
 \end{itemize}


\section{Collision Physics}

\subsection{Basic Principles}

\emph[fragile]{Basic Particle Simulation Algortihm}

\begin{algorithm}[H]
    \tiny
    %\DontPrintSemicolon
    \For {all particles}{
        sample position, direction, and velocity\;
        \While{particle is alive}{
             sample distance traveled\;
             move particle to new location\;
             \uIf {leak}{
                 particle is dead!\;
             }
             \uElseIf {crosses material boundary}{
                 move to boundary\;
             }
             \Else {
                 sample collision isotope\;
                 sample collision type\;
                 \uIf {absorption}{
                     particle is dead!\;
                 }
                 \Else{
                     sample direction and velocity\;
                 }
             }
        }
    }
\end{algorithm}



\emph{Source Sampling}
   \begin{itemize}
      \item Sample position, direction and velocity
      
      \item Depends on problem definition
      \begin{itemize}
        \item For fixed source problems, this is known
        \item For eigenvalue, this must be iterated
      \end{itemize}
      
      \item Fission sources are assumed to be isotropic and energy distribution are provided by tabulated pdfs, Watt's spectrum or other laws
      
      \item Watt Spectrum
      \begin{equation*}
          W(a,b,E') = Ce^{-aE'}sinh(\sqrt{bE'})
      \end{equation*}
      where $a$ and $b$ vary between isotopes
   \end{itemize}


\emph{Distance Traveled}
   \begin{itemize}
       \item Evaluate $\Sigma_t(E)$ of material in which particle is located
       
       \item Sample from the exponential pdf with parameter $\Sigma_{t}(E)$
       
       \item Calculate distance to nearest boundary
       
       \item If particle leaves medium, re-start sampling in new material with starting position at material boundary
       \begin{itemize}
            \item Exponential distribution has no memory
       \end{itemize}
       
       \item If distance is less than new material surface, particle makes a collision in medium
   \end{itemize}


\emph{Collision Isotope}
  Determine which isotope of interaction by sampling the discrete pmf 
  built from the $\Sigma_t$ of each isotope in the material:
  \begin{equation*}
       p_{j} = N_{j}\sigma_{tj}(E) / \Sigma_{t}(E)
  \end{equation*}
  where
  \begin{equation*}
        \Sigma_{t}(E) = \sum_{j} N_{j}\sigma_{tj}(E)
  \end{equation*}  
  and $j$ represents the isotope


\emph{Collision Type}
  Once the collision isotope is selected, determine which reaction will 
  occur by sampling a discrete pmf built from $\sigma$'s of this isotope
  \begin{equation*}
       p_{k} = \sigma_{k}(E) / \sigma_{t}(E)     
  \end{equation*}
  where $k$ is the reaction type and 
  \begin{equation*}
     \sigma_{t}(E) = \sigma_{n,n}(E) + \sigma_{n,n'}(E) + \sigma_{n,\gamma}(E) + \dots
  \end{equation*}


\emph{Collision Type}
If the collision is...
    \begin{itemize}
      \item \emph{absortion}: the particle is killed (in \underline{analog} simulations)
      \item \emph{scattering}: sample the outgoing energy and direction from the scattering data
      \item $\mathbf{(n,2n)}$: sample the outgoing energy and direction of each neutron...and then what?
      \item ...
   \end{itemize}


\emph{Scattering}
 \begin{itemize}
  \item Scattering laws are defined in either the CM or LAB system
  
  \item Simulation is performed in the LAB system
  
  \item Sample energy and direction (sometime they are correlated)
  
  \item \emph{ Example}: elastic scattering with target at rest
  \begin{equation*}
  E' = E \frac{(A^2+2A \mu_{CM}+1)}{(A+1)^2}
  \end{equation*}
  \begin{equation*}
  \mu_{L} = \frac{1+A \mu_{CM}}{\sqrt{A^2+2A \mu_{CM}+1}}
  \end{equation*}
 \end{itemize}


\emph{Exit Direction}
 \begin{itemize}
  \item The polar angle is given by the scattering law, $\mu_{L}$
  \item The azimuthal angle is sample uniformly over $2 \pi$
 \end{itemize}
  \begin{equation*}
    u' = \mu_{L}u+\frac{\sqrt{1-\mu_{L}^{2}}(uw \; cos(\phi)-v \; sin(\phi))}{\sqrt{1-w^2}}
  \end{equation*}
  \begin{equation*}
    v' = \mu_{L}v+\frac{\sqrt{1-\mu_{L}^{2}}(vw \; cos(\phi)+u \; sin(\phi))}{\sqrt{1-w^2}}
  \end{equation*}
  \begin{equation*}
    w' = \mu_{L}w+\sqrt{1-\mu_{L}^{2}}\sqrt{1-w^2} \; cos(\phi)
  \end{equation*}


\subsection{Additional Topics}

\emph{Secondary Particles/Fission Neutrons}
  \begin{itemize}
   \item One possibility
   \begin{itemize}
    \item Add fission neutrons only when fission reaction is selected
    \item Sample directly from $\nu(E)$
    \begin{equation*}
       n = int(\nu+\xi)
    \end{equation*}
   \end{itemize}
   
   \item Another possibility
     \begin{itemize}
        \item Add fission neutrons after each collision
        \item Sample from expected number of neutrons per collision
        \begin{equation*}
           r = \nu \sigma_{f}/\sigma_{t} \; \; \; n = int(r+\xi)
        \end{equation*}
     \end{itemize}
    
    \item Particles are ``banked'' and must be simulated before starting a new particle
  \end{itemize}


\emph{Woodcock Tracking}
  \begin{itemize}
   \item Basic particle tracking requires that new distance traveled be sampled everytime a material surface is crossed.
   \item Delta tracking provides a way to track particles across the entire geometry without stopping at material boundaries.
   \begin{itemize}
    \item Create a fictitous cross-section in all materials such that the total macroscopic cross-section is constant everywhere.
    \item When sampling a fictitous cross-section, nothing changes, just re-sample travel path.
   \end{itemize}
   \item Works well when $\Sigma_t$ is relatively uniform
  \end{itemize}


\emph[fragile]{Woodcock Tracking}

\begin{algorithm}[H]
    given location $x$, angle $\mu$, and $\Sigma_{\text{max}}$\;
    $s = 0$ \;
    \While{true}{
      $s = s -\ln{(\xi_1)}/\Sigma_{\text{max}}$\;
      \If {$\xi_2 < \Sigma_t(x+\mu s) / \Sigma_{\text{max}}$}{
        break\;
      }
    }
    return $s$\;
\end{algorithm}



% -----------------------------------------------------------------------------
%\section{Conclusions}
\emph{Recap}
  \begin{itemize}
   \item Random number sequences are not truly random
   \item LCG parameters must be chosen carefully: \emph{ know your RNG!!!}
   \item Continuous distributions can be sampled by either direct sampling, rejection sampling or discrete sampling (if discretized)
   \item Think like a neutron!
  \end{itemize}
  \vfill
  
  Exam \emph{ this} Thursday (9:30-11:00) \\
  Homework 4 due \emph{ next} Thursday


\emph{References}
 \begin{itemize}
  \item F. Brown, \emph{ Monte Carlo Lecture Notes} 
  \item I. Lux, L. Koblinger, \emph{ Monte Carlo Particle Transport Methods: Neutron and Photon Calculations}
  \item W. Dunn and K. Shultis, \emph{ Exploring Monte Carlo Methods}
 \end{itemize}

