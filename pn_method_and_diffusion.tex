
\chapter{P$\mathrm{_N}$ Method and Diffusion}

\begin{exercises}

  \item  \textbf{P$_{\mathbf{1}}$ Boundary Conditions}. Derive the $P_1$ equations for isotropic scattering and an isotropic source.  Derive the Marshak vacuum conditions for arbitrary left and right boundaries in a slab.  Do the same for the Mark conditions. 

  \item \textbf{P$_{\mathbf{2}}$ Equations}.Show for the case of linearly anisotropic scattering and isotropic source that the $P_2$ equations can be written as a second order partial differential equation similar in form to the typical neutron diffusion equation.

  \item \textbf{Numerical Solution of the P$_{\mathbf{1}}$ Equations}. Consider a slab of width 10 cm with $\Sigma_t = 1.0$ [cm$^{-1}$], and $c = \Sigma_s / \Sigma_t = 0.5$ (isotropic scattering in the lab system).  A uniform, isotropic source of $1$ [n/cm$^2$-s] is located in the first half of the slab, and both slab edges are subject to vacuum conditions. Write a code to solve the $P_1$ approximation to this problem using Marshak conditions.  Plot $\psi(x,\mu)$ at $x = 0$, $2.5$, $5.0$, $7.5$, and $10$ [cm].  Plot $\phi(x)$ over the whole slab.

  \item \textbf{Numerical Solution of the P$_{\mathbf{3}}$ Equations}. For the same problem, write a code to solve the $P_3$ approximation using Marshak conditions.

  \item \textbf{ Diffusion via asymptotics}.  Consider the following rescaling of the 1-d, mono-energetic transport equation with isotropic scattering. Finish me.

  \item \textbf{Legendre Addition Theorem}. Prove the Legendre polynomial addition theorem.  You may use any resource you want, but make sure you understand all steps of the proof.  A particularly straightforward approach begins as follows.  Start with an expansion of an arbitrary function in the full spherical harmonics.  Then, substitute the definition of the expansion coefficients back into the expansion.  Noting that the integral and summation can be switched, what function must the summation be?  

  \item \textbf{Defining the Scattering Angle}. Prove $\mu_0 = \cos(\theta)\cos(\theta')+\sin(\theta)\sin(\theta')\cos(\phi-\phi')$, where $\mu_0$ is the cosine of the scattering angle and ($\theta$,$\phi$) and $(\theta',\phi')$ are the original and final angles, respectively.  Include a diagram.

\end{exercises}
