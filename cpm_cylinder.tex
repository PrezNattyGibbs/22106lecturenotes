\chapter{CPM in Cylindrical Coordinates}
\label{lec:cpm_cylinder}

In this lecture, the collision probability method is derived for 
1-D cylindrical coordinates and specialized for the case of 
annular systems.  For many years, this approach was fundamental for 
pin-cell spectrum calculations.

\section*{Sample Code}

To be continued.

\section*{Further Reading}

Lewis and Miller. Hebert.  Carlvik 1964 paper.

%------------------------------------------------------------------------------%
\begin{exercises}

  %----------------------------------------------------------------------------%
  \item \textbf{White Boundary Conditions}. 
  
  Collision probabilities 
  found using the methods discussed in this lecture can be modified to 
  account for white boundary conditions by setting 
  \begin{equation*}
  R_i =  \Sigma_{i}V_i - \sum_{j} \Sigma_j V_j P_{j\to i} \, ,
  \end{equation*}
  and defining updated collision probabilities
  \begin{equation*}
  \Sigma_{i}V_i P^{\text{white}}_{j\to i} = 
    \Sigma_{i}V_i P_{j \to i} - \frac{R_i R_j}{\sum_k R_k} \, .
  \end{equation*}
  Prove this expression by using reciprocity relations.

    
\end{exercises}