\chapter{Iterative Solvers for Transport Equations}
\label{lec:iterative_solvers}

So far, three basic models for discretizing the transport equations have been
discussed.  The latter model, neutron diffusion, leads to a very 
straightforward system of linear equations for which standard techniques
readily apply.  The S$_{\text{N}}$ and MOC transport schemes also lead 
to a set of linear equations.  However, casting these equations in a 
standard form for solution by generic solvers is less obvious.   Because
our goal is ultimately to apply generic solvers to the equations, it
is useful here first to provide a general representation of the equations.

\subsection{Angular Flux Equations}

Before proceeding, we adopt an operator notation following Larsen 
and Morel \cite{larsen2010ado} to simplify
somewhat the exposition to follow.  
Recall that the scalar flux is defined
\begin{equation}
\begin{split}
  \phi(\vec{r}) &= \int_{4\pi} d \Omega \, 
                   \psi(\vec{r}, \Omega) \\
              &\approx \sum_{n = 1}^N w_n \psi_{n}(\vec{r}) \, .
\end{split}
\end{equation}
Introducing a \emph{discrete-to-moment} operator $\oper{D}$, we
enforce
\begin{equation}
 \phi = \oper{D} \psi \, ,
\end{equation}
where space and angle indices are implicit.  Moreover, we introduce
a \emph{moment-to-discrete} operator $\oper{M}$, defined
\begin{equation}
 \psi = \oper{M} \phi \, ,
\end{equation}
where in general
$\oper{D} \ne \oper{M}^{-1}$.  Finally,
defining the operator
\begin{equation}
 \oper{L} (\cdot) \equiv 
 \left ( \hat{\Omega} \cdot \nabla + 
   \Sigma_{t}(\vec{r}) \right ) (\cdot) \, ,
\end{equation}
the one group discretized transport equation becomes
\begin{equation}
  \oper{L}\psi = \oper{M}\oper{S}\phi + q \, ,
\label{eq:wgteop_psi}
\end{equation}
where $\oper{S}$ is the scattering operator (which can include
fission).  For the case of
isotropic scattering in one group, we have
$\oper{M} = \frac{1}{4\pi}$
while $\oper{S} = \Sigma_s(\vec{r})$.

For the multigroup problem, this notation generalizes to 
\begin{equation}
  \oper{L}_g\psi_g =  \oper{M} \sum_{g'=1}^G 
         \left (
           \oper{S}_{gg'} + 
           \frac{1}{k}\oper{X}_{g} \oper{F}_{g'} 
         \right )\phi_{g'} + q_g \, ,
\label{eq:mgteop_psi}
\end{equation}
where the fission source has been explicitly represented.  Here,
$\oper{X}$ represents the fission spectrum $\chi$ in operator form.

\subsection{Moments Equation}

\subsubsection{S$_{\text{N}}$ in Operator Form}

Equations \ref{eq:wgteop_psi} and \ref{eq:mgteop_psi} represent 
equations in which the unknown is the angular flux $\psi$.  In 
general, and particularly for reactor physics, our interest is in
computing reaction rates for which only the scalar flux is needed.  
Consequently, in practice the angular flux is rarely stored explicitly, 
but is rather computed on-the-fly during a sweep through the space-angle 
grid.  This
can be represented explicitly by manipulating the equations to be functions
only of the scalar flux (and higher order flux moments if scattering 
is anisotropic).  

To illustrate, consider Eq. \ref{eq:wgteop_psi}.  Multiplying
through by the space-angle \emph{transport sweep} 
operator $\oper{T} = \oper{DL}^{-1}$, we have
\begin{equation}
 \oper{D}\psi = 
    \oper{DL}^{-1} \oper{MS} \phi + \oper{DL}^{-1}q \, .
\end{equation}
Recalling $\phi = \oper{D}\psi$ and rearranging yields 
\begin{equation}
 (\oper{I} - \oper{TMS} ) \phi =  \oper{T}q \, ,
\label{eq:wgteop}
\end{equation}
or 
\begin{equation}
  \oper{A}_{\text{WG}} \phi = b \, ,
\end{equation}
where
\begin{equation}
 \oper{A}_{\text{WG}} \equiv \oper{I} - \oper{TMS}
\end{equation}
is the \emph{within-group transport operator} and
\begin{equation}
 b = \oper{T}q
\end{equation}
represents the \emph{uncollided} flux moments.  Eq. \ref{eq:wgteop} represents
a linear equation in standard form for the scalar flux moments $\phi$.  

The operator $\oper{DL}^{-1}$ has 
a very specific physical interpretation.  The inverse $\oper{L}^{-1}$
represents the set of space-angle sweeps from one global boundary 
to another.  The additional factor  $\oper{D}$ implies that along
the space-angle sweep, $\phi$ is updated, thus eliminating the 
need to store $\psi$.  This is exactly how S$_{\text{N}}$ and MOC solvers 
have traditionally been implemented.  

The extension to multigroup yields a similar form
\begin{equation}
   \left ( \oper{I} - \oper{T_{\text{MG}}M_{\text{MG}}} 
      \left ( \oper{S}_{\text{MG}} + 
              \frac{1}{k}\oper{X}_{\text{MG}} \oper{F}_{\text{MG}}^{\transp} 
      \right ) 
   \right ) \phi 
   =  \oper{T_{\text{MG}}}q_{\text{MG}} \, ,
\label{eq:mgteop}
\end{equation}
where $\oper{T_{\text{MG}}}$  and $\oper{M_{\text{MG}}}$ are  block 
diagonal operators
having blocks 
$\oper{T}_g$ and $\oper{M}$, 
respectively, 
$\oper{X_{\text{MG}}}$, $\oper{F_{\text{MG}}}$, and $q_{\text{MG}}$ 
are the 
column vectors having elements 
$\oper{X}_g$, $\oper{F}_g$, and $q_g$
respectively, and the multigroup scattering operator is defined
\begin{equation}
       \oper{S_{\text{MG}}} = 
           \left(
           \begin{array}{ccc}
              \oper{S}_{11} & \cdots & \oper{S}_{1G} \\
              \vdots           & \ddots & \vdots          \\
              \oper{S}_{G1} & \cdot   & \oper{S}_{GG}
           \end{array} 
           \right ) \, .
\label{eq:mgscatter}
\end{equation}
Similar to the within-group problem, we define the \emph{multigroup
transport operator} to be
\begin{equation}
 \oper{A}_{\text{MG}} =  \oper{I} - \oper{T_{\text{MG}}M_{\text{MG}}} 
      \left ( \oper{S}_{\text{MG}} + 
              \frac{1}{k}\oper{X}_{\text{MG}} \oper{F}_{\text{MG}}^{\transp} 
      \right ) \, .
 \label{eq:mgto}
\end{equation}

\subsubsection{MOC in Operator Form}

While the operators and equations apply directly the S$_{\text{N}}$ method,
slight changes must be made to represent MOC. In MOC, we iterate on the 
quantities averaged over a \emph{cell},
specifically the scalar flux moments $\phi_{\text{cell}}$.  However, the
angular fluxes are computed for each track \emph{segment}, and each cell 
can have several tracks crossing it.  The average angular flux on each 
track segment in a given cell is used to generate the average scalar flux.

This can be cast in operator form for a one group problem as
\begin{equation}
  \oper{L}\psi_{\text{seg}} = 
    \oper{P M S}\phi_{\text{cell}} +
      \oper{P} q_{\text{cell}} \, ,
    \label{eq:traneq}
\end{equation}
where $q_{\mathrm{cell}}$ is the cell-wise external source
and $\oper{P}$ is a cell-to-segment operator, ``P'' denoting prolongation,
though not quite in the multigrid sense.  

We can also define a segment-to-cell operator (restriction) 
$\oper{R}$ so that
\begin{equation}
   \psi_{\text{cell}} = \oper{R}\psi_{\text{seg}} \, ,
\end{equation}
and
\begin{equation}
    \phi_{\text{cell}} = \oper{DR}\psi_{\text{seg}} \, . 
    \label{eq:psitophi}
\end{equation}
Finally, the moments equation akin to Eq. \ref{eq:wgteop} is
\begin{equation}
   \left ( \oper{I} -  \oper{DRL}^{-1}\oper{PMS} \right ) 
     \phi_{\text{cell}} =  
  \oper{DRL}^{-1}\oper{P} q_{\text{cell}}  \, .
\end{equation}
In the context of MOC, the operator
\begin{equation}
 \oper{A}_{\text{WG-MOC}} =  \oper{I} -  \oper{DRL}^{-1}\oper{PMS} 
\end{equation}
is implied when discussing the within-group transport operator, and 
similarly for the multigroup variant.

\subsection{Boundary Conditions}

So far, we have neglected to treat boundary conditions.  
In particular, reflecting (or periodic, or white) boundary conditions 
create an additional set of unknowns, namely the incident 
(or exiting) boundary fluxes.  In theory, treating these conditions is
straightforward, but we refrain from further discussion for two 
reasons. First, adding boundary unknowns complicates presentation of 
the algorithms discussed below.  Second, and more importantly,
response function calculations use vacuum conditions 
only, and so treatment of reflecting boundaries is outside our scope.
A more detailed discussion can be found in the work 
of Warsa \emph{et al.} \cite{warsa2004kim}.

\section{Transport Solvers}

Consider the fixed source multigroup transport problem
represented by
\begin{equation}
 \oper{A}_{\text{MG}} \phi = b \, .
\end{equation}
The traditional method for solving the multigroup equations is via 
a nested iteration in which a series of within-group equations is 
solved for each group, with the scattering (and possibly) fission 
sources updated between each group solve.  More modern treatments
view the equations as a complete set to be solved simultaneously.


\subsection{Gauss-Seidel}

The Gauss-Seidel method has long been used to solve the multigroup 
equations.  For standard problems, the method first solves the 
fast (within-)group equation.  The updated fast group flux can then be used 
to define the in-scatter source for the next group, and so on.  Because
this algorithm really implies inversion of each group \emph{block} (as 
defined in Eq. \ref{eq:mgteop}), the method is more accurately described 
as \emph{block} Gauss-Seidel in energy.


Algorithm \ref{alg:gauss_seidel_energy} describes a basic implementation 
of the method.  Historically, convergence of the method is based on the 
difference of successive fluxes, either as an absolute norm as indicated 
or a norm of the relative point-wise difference.  Assessing convergence 
in this way can be misleading.  The difference between successive
iterates can be much smaller than the difference between an iterate
and the solution, sometimes resulting in premature convergence.  Section 
\ref{sec:comparison_of_norms} provides a numerical comparison of various
convergence criteria.

\begin{algorithm}[h]
  \SetCommentSty{small}
  \DontPrintSemicolon
  \KwData{initial guess for group fluxes $\phi^{(0)}$, 
          external source $q$, 
          eigenvalue $k$, 
          tolerance $\tau$,
          maximum number of iterations $N$}
  \KwResult{converged group fluxes $\phi$}
  $n = 1$ \;
  \While{$\tau > \epsilon$ and $n < N$}
  {
    \For{$g$ from 1 to $G$}
    {
      \tcp{Compute in-scatter and in-fission sources}
      $b_g = \oper{T}_g \left(  \oper{M} \sum_{g'=1}^G 
              \left ( \oper{S}_{gg'} + \frac{1}{k}\oper{X}_{g} \oper{F}_{g'} 
              \right )\phi^{(n-1)}_{g'} + q_g \right )$ \;
      \tcp{Solve the within-group problem}
      $\phi^{(n)}_{g} = \oper{A}^{-1}_g b_g$ \;
    }
    $\tau = ||\phi^{(n)}-\phi^{(n-1)}||$\;
    $n = n + 1$\;
    
  }
  \caption{Gauss-Seidel Algorithm for the Multigroup Transport Equation}
  \label{alg:gauss_seidel_energy}
\end{algorithm}

For problems in which there is no upscatter and no fission, Gauss-Seidel
is an essentially exact scheme assuming the within-group equations are
solved exactly.  However, for cases with upscatter or fission, 
extra ``upscatter'' iterations are required, and in some cases, 
the convergence of Gauss-Seidel becomes prohibitively slow.

\subsection{Source Iteration}
\index{source iteration}
\index{Richardson iteration|see{source iteration}}
The standard method for solving the within-group transport equation
has been \emph{source iteration}. The basic idea is a simple one:
given an external source (including in-scatter and fission), we guess the 
flux, compute the within-group scattering source, solve for a new flux, and 
repeat until converged.  Mathematically, source iteration is 
defined by the process 
\begin{equation}
 \phi^{(n)} = \oper{TMS}\phi^{(n-1)} + \oper{T}q \, .
\end{equation}
However, recall that Richardson iteration for the system $\oper{A}x=b$ 
is just 
\begin{equation}
 x^{(n)} = (\oper{I}-\oper{A})x^{(n-1)} + b \, .
\end{equation}
Since $\oper{I} - \oper{A}_{\text{WG}} = \oper{TMS}$, we see that 
source iteration is equivalent to Richardson iteration.

Source iteration has a particularly intuitive physical interpretation.
Suppose our initial guess is zero.  The resulting flux is just the 
right hand side, or the uncollided flux.  Performing one iteration 
adds the contribution of neutrons having undergone a single collision.
A second iteration includes the twice-collided neutrons, and so on. 
For systems in which neutrons undergo many scattering events, the physics 
suggests the process can be painfully slow to converge.  The math verifies
the physics \cite{larsen2010ado}: the error $||\phi^{(n)}-\phi^{(n-1)}||$ of 
the infinite medium one group problem goes as the scattering 
ratio $c = \frac{\Sigma_s}{\Sigma_t}$, meaning that increased scattering
leads to slower convergence.

\subsection{Krylov Solvers}
\label{sec:krylovsolvers}

Because of the limitations of the Gauss-Seidel and source iteration
schemes, much work has been done to apply modern linear solvers to
transport problems.  One of the most successful class of solvers 
studied consists of Krylov subspace methods.  

\subsubsection{Overview}

One can categorize linear (and eigenvalue) solvers as being 
\emph{stationary} or \emph{nonstationary}.  Stationary methods
produce updated solutions using only a single previous solution, and
the Gauss-Seidel and Richardson methods are both stationary.  Other 
well-known examples include the Jacobi and 
successive over-relaxation (SOR) methods.
On the other hand, nonstationary methods produce a solution based on 
two or more previous iterates (or the information used to create those 
iterates). 

Krylov methods are nonstationary methods that rely on 
construction of a so-called \emph{Krylov subspace} of dimension $n$, 
defined for an $m \times m$ operator $\oper{A}$ as
\begin{equation}
 \mathcal{K}(n,x_0) \equiv \text{span} 
     \{x_0,\, \oper{A}x_0,\, \oper{A}^2x_0,\, 
         ,\ldots,\,, \oper{A}^{m-1}x_0 \} \, , 
 \label{eq:krylovsubspace}
\end{equation}
for some initial, possibly random, vector $x_0$.  The main idea of Krylov
subspace methods is to find  $x \in \mathcal{K}(m,x_0)$  
 that ``best'' solves the system of interest, be it an 
eigenproblem or linear system, where it is assumed that $m \ll n$.

Working with $\mathcal{K}(n, x_0)$ directly is difficult numerically, since
repeated application of $\oper{A}$ sends the initial vector $x_0$ into the
same direction, namely that of the dominant eigenvector of $\oper{A}$.  
Hence, the basis 
must be orthogonalized. The canonical approach for nonsymmetric 
operators is Arnoldi's method, which by successive application of the 
modified Gram-Schmidt process yields the Arnoldi decomposition
\begin{equation}
 \oper{A}\oper{V} = \oper{V} \oper{H} + fe^{\transp}_m \, ,
\end{equation}
where $\mathbf{V} \in \mathbb{R}^{m\times n}$  consists of
orthonormal columns,
$\mathbf{H} \in \mathbb{R}^{n \times n} $ is an upper Hessenberg matrix, 
$e_n$ is the
$n$-vector of all zeros except a one in the $n$th location, and $f$ is 
the residual, which is orthogonal to the columns of $\mathbf{V}$.  

The most popular Krylov method for nonsymmetric linear systems (such as the 
transport equations above) is GMRES \cite{saad1986gmr}.  The basic idea 
of GMRES is straightforward:  the $n$th step of GMRES produces a 
$n\times n$ Hessenberg matrix and the corresponding basis $\oper{V}$,
and the approximate solution $x_n$ is found by minimizing the residual 
norm $||r||_2 = ||\oper{A}x_n-b||_2$
for $x_i \in \mathcal{K}(n, x_0)$.  In other words, $x_n$ must be in 
the column space of $\mathcal{K}(n, x_0)$, mathematically 
expressed as $x_n = \oper{V} y$,  where 
$y$ satisfies 
\begin{equation}
\begin{split}
  ||\oper{AV}y-b||_2 &= ||\oper{V}^{\transp} \oper{A} \oper{V} x_n -\oper{V}^{\transp} b|| \\
                     &= || \oper{H} y - \oper{V}^{\transp} b || \, .
\end{split}
\end{equation}
The last equation shows that GMRES can be thought of as finding the 
best solution $x_n \in \mathcal{K}(n, x_0)$ in a least-squares sense.

\subsubsection{Preconditioning}

While Krylov methods are generally more robust than the classical 
stationary methods, their performance can be improved significantly
via \emph{preconditioning}.  A preconditioner $\oper{M}$ is an operator 
whose inverse satisfies $\oper{M}^{-1} \approx \oper{A}^{-1}$ in some sense 
and is relatively inexpensive to apply.

A \emph{left preconditioned} linear system is
\begin{equation}
  \oper{M}^{-1}\oper{A}x = \oper{M}^{-1}b
\end{equation}
while a \emph{right preconditioned} system is 
\begin{equation}
  \oper{A}\oper{M}^{-1} y = b
\end{equation}
with $x = \oper{M}^{-1} y$.  The left preconditioned residual differs 
from the original residual but may be a better metric for convergence. 
The right preconditioned system preserves the original residual.  

A preconditioner typically leads to a clustering of eigenvalues.  As 
an extreme example, suppose that $\oper{M} = \oper{A}$.  The 
preconditioned operator is then $\oper{AM}^{-1} = \oper{I}$, for 
which all the eigenvalues are unity.  Of course, to apply $\oper{M}^{-1}$ 
in this case represents solving the original problem.  While preconditioners
cannot in general be expected to yield a set of eigenvalues equal to unity, 
any clustering typically improves convergence.  Often, even pushing 
eigenvalues away from the origin tends to improve 
convergence \cite{larsen2010ado}.
Chapter \ref{chp:transport_pc} provides a relatively thorough 
development of several diffusion-based preconditioners for the 
transport equation.

\subsubsection{Krylov Methods for the Transport Problems}

Krylov solvers have been used 
to solve both the within-group \cite{warsa2004kim} and multigroup
\cite{evans2010tdf} transport equations.  For the multigroup 
equations in particular, the independent nature of the group-wise 
blocks makes parallelization in energy much more straightforward
than for Gauss-Seidel.

For problems in which there is no fission and 
upscatter is limited to a subset of thermal groups, it is
possible to solve the downscatter groups via Gauss-Seidel
and to use a Krylov method on the thermal block of Eq. \ref{eq:mgteop}.
Doing so can yield improved efficiency for some problems \cite{evans2010tdf}.  
However, when fission is included, as is often true for response 
function generation, there is always
thermal-to-fast coupling, and solving the full system via
a Krylov method is to be preferred.

