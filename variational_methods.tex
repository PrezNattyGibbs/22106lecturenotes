\chapter{Variational Methods}

In this lecture, we investigate use of \textit{variational principles} in the context of nuclear engineering.  An entire course could be devoted to this subject; here, we focus on the same two quantities analyzed in Lecture \ref{lec:adjoint}, namely the reaction rate for a fixed source problem and the eigenvalue of an eigenvalue problem.  We demonstrate that variational principles can be used to estimate these quantities using approximate inputs (i.e. fluxes).  We finish by showing how our first order perturbation theory can be derived directly from variational principles.

\section*{Variational Principles, Functionals, and Stationary Points}

A variational principle casts a particular function, usually a problem's solution, as the stationary point of some \textit{functional}. Often, the functional itself is called the variational principle.  A functional is simply a function that takes another function as its argument and returns a scalar as its value.  Consider a function $f(x) = Ax + B$.  A possible functional would be $F[f(x)] = \int_{x_1}^{x_2} f(x)dx$, which certainly yields a scalar value.  Typically, the value of the functional represents a quantity of interest (e.g. a reaction rate).

We see that functionals are quite like functions; how exactly then do we define a \textit{stationary point}, and what does it mean in the context of a variational principle?  Recall from elementary calculus that a stationary point is the value of the independent variables such that the function reaches a local extremum (or saddle-point), i.e. the function's derivative (or gradient) vanishes.  The same idea applies to functionals.  Defining the ``weak derivative'' of $F$ at a point $f(x)$ in the direction $g(x)$ as
\begin{equation}
 \delta F[f,g] = \lim_{\epsilon \to 0} = \frac{ F[f+\epsilon g] - F[f] }{\epsilon} \, ,
\end{equation}
the \textit{first variation} of $F$ is defined 
\begin{equation}
 \delta F[f,g] = \Big ( \frac{d}{d\epsilon} F[f+\epsilon g] \Big )\Big |_{\epsilon = 0} \, ,
\end{equation}
for arbitrary $g$.  When $\delta F[\tilde{f},g] = 0$ for all $g$, $\tilde{f}$ is called a stationary point of $F$, and the expression
\begin{equation}
 \delta F[\tilde{f},g] = 0 \, ,  \, \, \, \, \, \, \forall g 
\end{equation}
defines the variational principle for $\tilde{f}$.  For $F[f]$ that represents a quantity of interest, $F[\tilde{f}]$ represents the true value for that quantity.  Moreover, very near the stationary point, $\delta F \to 0$ by construction, and the errors in $F$ (and hence the quantity of interest) are second order, which gives rise to the utility of variational approximations.\footnote{When our interest is a quantity (e.g. a reaction rate) rather than the solution (e.g. a flux), we often refer to $F[f]$ as a variational principle for that quantity of interest; in the next sections, we look at variational principles for the reaction rate and eigenvalue.}

\section*{A Simple, Illustrative Example}

It is easiest to understand these ideas through a simple example (unrelated to nuclear engineering).  Suppose we wish to find the curve giving us the shortest difference between two points in a plane.  Of course, this is intuitive: the solution should be a line.  We show this using variational techniques.  

Let the curve be $y(x)$; from any calculus book, we know the differential arc length is then
\begin{equation}
 dl = \sqrt{ dx^2 + dy^2 } = \sqrt{1 + (y')^2} \, .
\end{equation}
We take as our functional the arc length,
\begin{equation}
 F[y] = \int^{x_2}_{x_1} \sqrt{1 + (y')^2} dx \, ,
\end{equation}
where ($x_1$,$y_1$) and ($x_2$,$y_2$) are the points of interest.  

Taking the first variation of $F$,
\begin{equation}
\begin{split}
\delta F[y,g] &= \Big ( \frac{d}{d\epsilon}\int^{x_2}_{x_1} \sqrt{1 + (y'+\epsilon g')^2} dx  \Big ) \Big |_{\epsilon=0} \\
              &=  \Bigg ( \int^{x_2}_{x_1} \frac{(y'+\epsilon g')g'}{\sqrt{1 + (y'+\epsilon g')^2}}  \Bigg ) \Bigg |_{\epsilon=0} \\
              &= \int^{x_2}_{x_1} \frac{(y')g'}{\sqrt{1 + (y')^2}}  \, ,
\end{split}
\end{equation}
we find that the arc length is minimized when
\begin{equation}
\delta F[y,g] = 0 =  \int^{x_2}_{x_1} \frac{(y')g'}{\sqrt{1 + (y')^2}}  \, .
\label{eq:exampleprinciple}
\end{equation}
By inspection, the variational principle is satisfied for arbitrary $g$ only if $y' = 0$.  This relation is known as the \textit{Euler equation}\footnote{In general, setting the first variation to zero gives rise to a set of partial differential equations collectively known as the Euler equations.}.  Of course, to satisfy $y'=0$ requires our solution $\tilde{y}(x)$ to be of the form $Ax+B$, i.e. linear, as expected.

\section*{A Variational Principle for Fixed Source Problems}

Suppose we are interested in a linear functional of the flux, such as $G_{fs}[\psi] = \langle \psi, \Sigma_d \rangle$, a reaction rate. An appropriate variational principal is represented by the \textit{generalized Roussopolos functional}
\begin{equation}
 F_{fs} [\psi,\psi^*] = G_{fs}[\psi] + \langle \psi^*, (Q-L\psi) \rangle \, .
\end{equation}
Note, $\delta F_{fs}[\psi,\psi^*] = 0$ is a variational principle for $G_{fs}$ if the corresponding solution $\psi$ yields $F_{fs}[\psi,\psi^*] = G_{fs}[\psi]$.  We see this is so when the second term of $F$ vanishes when $\psi$ solves $L\psi = Q$, i.e. when $\psi$ is the solution to the transport equation.

To determine the variational principle, we take the first variation of $F_{fs}$,
\begin{equation}
 \begin{split}
  \delta F_{fs} [\phi,\phi^*] &= \Bigg ( \frac{d}{d\epsilon} \Big ( \langle (\phi+\epsilon \delta \phi) \Sigma_d \rangle + \langle (\phi^* + \epsilon \delta \phi^*), (Q-L(\phi+\epsilon \delta \phi)) \rangle \Big ) \Bigg ) \Bigg|_{\epsilon=0} \\
   &= \langle \delta \phi, \Sigma_d \rangle - \langle \phi^*,L \delta \phi \rangle + \langle \delta \phi^*, Q \rangle - \langle \delta \phi^*, L \phi \rangle \, .
 \end{split}
\end{equation}
Noting $\langle \phi^*,L\delta \phi \rangle = \langle L^* \phi^*,\delta \phi \rangle$, we have for our principle 
\begin{equation}
 0 = \langle \delta \phi, (\Sigma_d - L^* \phi^*) \rangle + \langle \delta \phi^*, (Q-L\phi) \rangle \, ,
\end{equation}
which is satisfied when $L\phi = Q$ and $L^* \phi^* = \Sigma_d$.  These are the corresponding Euler equations, and we see they are just the original forward and adjoint transport equations.

The importance of this variational principle (and others in general) is that it gives us an estimate for $G_{fs}$ accurate to second order for approximate fluxes $\phi$ and $\phi^*$.  To demonstrate this, suppose the true solutions to the Euler equations are $\psi$ and $\psi^*$.  Let $\phi = \psi + \delta \psi$ and $\phi^* = \psi^* + \delta \psi^*$.  We subsitute these expressions into $F$ and find
\begin{equation}
 \begin{split}
    F_{fs} [\phi,\phi^*] &= \langle \Sigma_d, (\psi+\delta \psi) \rangle + \langle (\psi^*+\delta \psi^*),(Q-L(\psi+\delta \psi^*) \rangle \\
    &= \langle \Sigma_d,\psi \rangle + \langle \Sigma_d, \delta \psi \rangle + \langle \psi^*,Q \rangle + \langle \delta \psi,Q \rangle  \\ 
    &- \langle \psi^*, L\psi \rangle - \langle \psi^*, L\delta \psi \rangle - \langle \delta \psi^*, L\psi \rangle - \langle \delta \psi^*, L\delta \psi \rangle \\
    &= \langle \Sigma_d, \psi \rangle - \langle \delta \psi^*, L\delta \psi \rangle \\
    &= G_{fs}[\psi] + \theta(\delta^2) \, .
 \end{split}
 \label{eq:secondorderfs}
\end{equation}
Hence, $F$ provides a first order accurate (i.e. good through first order) estimate of the reaction rate given approximate forward and adjoint fluxes.

\section*{A Variational Principle for Eigenvalue Problems}

For eigenvalue problems, an appropriate functional is the \textit{Rayleigh quotient}
\begin{equation}
 F_{ev}[\psi,\psi^*] = \frac{\langle \psi^*,L\psi \rangle}{ \langle \psi^*, F \psi \rangle } \, .
 \label{eq:rayleigh}
\end{equation}
Here, the quantity of interest is $G_{fs} = \lambda$, and, unlike the fixed source case, we have $G_{fs} = F_{ev}$.  You should show that $F_{ev}$ is in fact a valid expression for $\lambda$.  The associated Euler equations are just the forward and adjoint eigenvalue equations, which can be shown by setting the first variation of $F_{ev}$ to zero, an exercise left to the reader.

For approximate fluxes $\phi$ and $\phi$, it can be shown that
\begin{equation}
 F_{ev}[\phi,\phi^*] = \lambda + \theta(\delta^2) \, ,
\end{equation}
the proof of which is left as an exercise.

\section*{Perturbation Theory from Variational Principles}

In general, it is possible to find first order perturbation estimates using the expression
\begin{equation}
 \delta = \bar{F}[\psi,\psi^*] - F[\psi,\psi^*] \, ,
 \label{eq:pertfromvar}
\end{equation}
where $F$ is the appropriate functional with nominal operators and $\bar{F}$ is the function evaluated with perturbed operators.  It must be stressed that $\psi$ and $\psi^*$ are assumed to be the exact fluxes for the unperturbed problem.

As an example, we re-derive the first order perturbation to a detector response.  Suppose our perturbations to our system include $L+\delta L$, $Q+\delta Q$, and $\Sigma_d + \delta \Sigma_d$.  Then Eq. \ref{eq:pertfromvar}  gives
\begin{equation}
\begin{split}
 \delta_{fs} &= \bar{F}[\psi,\psi^*] - F[\psi,\psi^*] \\
        &= \langle \psi, (\Sigma_d+\delta \Sigma_d) \rangle + \langle \psi^*,(Q+\delta Q-(L+\delta L)\psi ) \rangle - \langle \psi, \delta \Sigma_d \rangle \\
        &= \langle \psi,\delta \Sigma_d \rangle + \langle \psi^* , \delta Q \rangle - \langle \psi^*,\delta L \psi \rangle \, .
\end{split}
\label{eq:fixedpertvar}
\end{equation}
The last line is equivalent to Eq. \ref{eq:fixedpertresp} with the addition of the first term, which explicitly accounts for changes in the detector response function.

Eq. \ref{eq:pertfromvar} can also be applied to the Rayleigh quotient, yielding Eq. \ref{eq:eigenpertresp}, proof of which is also left as an exercise.

\section*{Further Reading}

 Most of the material presented here is contained in Duderstadt and Martin \cite{duderstadt1976tt} Chapter 7 in addition to Chapter 1 of Stacey \cite{stacey1974vmn}.  The former contains several examples and a variational derivation of the diffusion equation.  The reader is also encouraged to look up Pomraning's large body of work on variational methods\footnote{It is worth noting that Stacey and Pomraning, both with prolific work in variational methods, did their graduate work in this department.}.

\begin{exercises}

  \item \textbf{Second Order Accuracy}. Show that the Roussopoulos functional $F[\phi,\phi^*]$ gives a second order accurate value for $\langle \psi,\Sigma_d \rangle$ when $\phi$ and $\phi^*$ are approximate values of $\psi$ and $\psi^*$, i.e. prove Eq. \ref{eq:secondorderfs}.

  \item \textbf{Fixed Source Perturbation}. Prove Eq. \ref{eq:fixedpertvar}.
 
  \item \textbf{Rayleigh Quotient}. (a) Take the first variation of the Rayleigh quotient with respect to the forward and adjoint fluxes, and show the stationarity conditions are just the forward and adjoint eigenvalue problems.  (b) Demonstrate that the Rayleigh quotient is a second order estimator for $\lambda$.

  \item \textbf{Losses-to-Gains}. Directly from the eigenvalue equation, we find 
  \begin{equation*}
    \lambda = (L\psi)/(F\psi) \, ,
  \end{equation*}
  i.e. losses-to-gains.  Show why this is not a variational principle for $\lambda$.  Consider again Eq. \ref{eq:rayleigh}.  How does the adjoint change the physical interpretation of gains-to-losses?

  \item  \textbf{Eigenvalue Perturbation}. Prove that $\bar{F}_R[\psi,\psi^+]-F_R[\psi,\psi^+]$ yields the first order perturbation estimate for $\delta \lambda$ given by Eq. \ref{eq:eigenpertresp}.

  \item  \textbf{Applying Roussopolos}. Consider a 1-d, 1-speed diffusion problem in a slab of width $2a$, $\Sigma_a = 0.022$, and $D = 0.14$. (a) Solve for the exact scalar flux, using the conditions $\phi(\pm a)=0$ (neglect extrapolated boundaries), assuming a uniform source $Q(x) = 1$ and $a=1$.  (b) Compute the total absorption rate in the slab.  (c) Now, approximate the solution as $\tilde{\phi}(x) = Ax^2 + Bx +C$ with the same maximum (at $x=0$) and the same boundary conditions.  Compute the absorption rate  using $\tilde{\phi}(x)$. (d) Now, recalling the 1-d, 1-speed diffusion equation is self-adjoint, use the Roussopolos principle to compute the absorption rate.  What can you conclude? 

  \item \textbf{More Roussopolos} For same problem, estimate the total absorption rate if something causes $\Sigma_a = 1.1$.  Be careful to consider the impact of the change in $\Sigma_a$ on both $\Sigma_d$ and $L$.

\end{exercises}